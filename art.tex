\documentclass[12pt]{article}

\usepackage{graphicx}
%\pagestyle{empty}
\usepackage{dsfont}
\usepackage{textcomp}
\usepackage{amsmath}
\usepackage{courier}
\usepackage[utf8]{inputenc}
\usepackage{graphicx}
\graphicspath{{images/}}
\usepackage{amsthm}
\usepackage{eucal}
\usepackage{amssymb}
%\usepackage{mdsymbol}
\usepackage{mathrsfs}
\usepackage{dsfont}
\usepackage[usenames,dvipsnames,svgnames,table]{xcolor}
\usepackage{caption}
\usepackage{tikz}
\usepackage{tikz-cd}
\usepackage{wrapfig,lipsum,booktabs}
\usepackage{caption}
\usepackage[square, longnamesfirst,numbers]{natbib}
\usepackage[ruled,linesnumbered]{algorithm2e}
\usepackage{todonotes}
\usepackage[all]{nowidow}

%\usepackage{hyperref}
\usepackage{amstext} % for \text macro
\usepackage{array}   % for \newcolumntype macro
\usepackage[unicode]{hyperref}
\usepackage{verbatim}
\usepackage{makecell}
\usepackage{float}
\usepackage{mathtools}

\newcolumntype{C}{>{$}c<{$}} % math-mode version of "l" column type

\relpenalty=10000
\binoppenalty=10000

\newcommand{\den}{\mathop{\mathgroup\symoperators den}\nolimits}
\newcommand{\ggd}{\mathop{\mathgroup\symoperators ggd}\nolimits}
\newcommand{\des}{\text{d.e.s.d.a.\ }}
%\newcommand{\opg}{\text{d.e.s.d.a.\ }}
\newcommand{\blokje}{\hfill $\Box$\\}
%Om je bewijs of je antwoord af te sluiten.
\newcommand{\header}[1]{\vspace{0.5cm}\framebox[\linewidth]{\textsc{Exercise #1}}\\}
\newcommand{\deel}[1]{\textbf{#1)}\ }
\newcommand{\macht}[1]{\mathcal{P}(#1)}
\newcommand{\entier}[1]{\left\lfloor #1 \right\rfloor}
\newcommand{\A}{\mathbb{A}}
\newcommand{\N}{\mathbb{N}}
\newcommand{\Z}{\mathbb{Z}}
\renewcommand{\G}{\mathbb{G}}
\renewcommand{\C}{\mathbb{C}}
\newcommand{\Q}{\mathbb{Q}}
\newcommand{\va}{\mathbf{a}}
\newcommand{\vx}{\mathbf{x}}
\newcommand{\vy}{\mathbf{y}}
\newcommand{\vz}{\mathbf{z}}
\newcommand{\Fcal}{\mathcal{F}}
\newcommand{\Lcal}{\mathcal{L}}
\renewcommand{\O}{\mathcal{O}}
\renewcommand{\P}{\mathbb{P}}
\newcommand{\Pcal}{\mathcal{P}}
\newcommand{\NP}{\mathcal{NP}}
\newcommand{\NPC}{\mathcal{NPC}}
\newcommand{\F}{\mathbb{F}}
\renewcommand{\angle}[1]{\hspace{-2pt}\left\langle #1 \right\rangle}
\newcommand*\diff{\mathop{}\!\mathrm{d}}
\newcommand{\rad}{\text{rad}}
\newcommand{\x}{x}
\newcommand{\m}{\mathfrak{m}}
\newcommand{\tensor}{\otimes}
\newcommand{\Ring}{\textbf{Rings}}
\newcommand{\Sets}{\textbf{Sets}}

\DeclareMathOperator{\mon}{mon}
\DeclareMathOperator{\Hom}{Hom}
\DeclareMathOperator{\Gal}{Gal}
\DeclareMathOperator{\End}{End}
\DeclareMathOperator{\sgn}{sgn}
\DeclareMathOperator{\trdeg}{trdeg}
\DeclareMathOperator{\rk}{rk}
\DeclareMathOperator{\Fun}{Fun}
\DeclareMathOperator{\im}{im}
\DeclareMathOperator{\id}{id}
\DeclareMathOperator{\tr}{tr}
\DeclareMathOperator{\sm}{sm}
\DeclareMathOperator{\pr}{pr}
\DeclareMathOperator{\diag}{diag}
\DeclareMathOperator{\Spec}{Spec}
\DeclareMathOperator{\MaxSpec}{MaxSpec}
\DeclareMathOperator{\Spf}{Spf}
\DeclareMathOperator{\sep}{sep}
\DeclareMathOperator{\Ann}{Ann}
\DeclareMathOperator{\GL}{GL}
\DeclareMathOperator{\Mat}{Mat}
\DeclareMathOperator{\Jac}{Jac}
\let\S\undefined
\DeclareMathOperator{\S}{S}
\let\div\undefined
\DeclareMathOperator{\div}{div}

\frenchspacing

\begingroup
\makeatletter
\@for\theoremstyle:=definition,remark,plain\do{%
	\expandafter\g@addto@macro\csname th@\theoremstyle\endcsname{%
		\addtolength{\thm@preskip}{5pt}
		\setlength{\thm@postskip}{10pt}
	}%
}
\endgroup %Zorgt voor mooie enters voor claims


\title{A geometric approach to linear Chabauty}
\author{Pim Spelier}
\date{November 2015}

%\newtheoremstyle{mytheoremstyle}{5pt}{-15pt}{\itshape}{}{\bfseries}{.}{.5em}{} 

\theoremstyle{plain}
\newtheorem{thm}{Theorem}[section] % reset theorem numbering for each section
\newtheorem{lem}[thm]{Lemma} % same for lemma's
\newtheorem{conj}[thm]{Conjecture} % same for vermoedens
\newtheorem{cor}[thm]{Corollary} % same for vermoedens
\newtheorem{prop}[thm]{Proposition} % same for vermoedens
\newtheorem{algo}[thm]{Algorithm} % same for vermoedens

\theoremstyle{definition}
\newtheorem{defn}[thm]{Definition} % definition numbers are dependent on theorem numbers
\newtheorem{exmp}[thm]{Example} % same for example numbers

\theoremstyle{remark}
\newtheorem{rem}[thm]{Remark} % remark numbers are dependent on theorem numbers

\usepackage{chngcntr}
\counterwithin{table}{section}

\newcommand{\overbar}[1]{\overline{#1}}

%\setlength{\parindent}{0pt}
%\setlength{\parskip}{\baselineskip}

\newcommand\legendre[2]{
  \genfrac(){}{}{#1}{#2}
}

\newcommand{\ord}{\operatorname{ord}}

\begin{document}
%\nocite{*}

\vspace*{1em}

\begin{center}

{\Large\bf 
P.\ Spelier, M.Sc\\
prof. dr. S.J. Edixhoven
}

{\sc 
Leiden University, Mathematical Institute\\
} 

\vspace{2em} 

{\LARGE\bf 
A geometric approach to linear Chabauty
} 
\vspace{2em}

{\large\bf 
\today
}

\end{center}

\listoftodos
\todo{Bas, zou je wellicht een bewijs kunnen schrijven dat onder de voorwaarde $r < g$ (ik weet niet of er nog andere nodig zijn) de verzameling gedefinieerd in Theorem~\ref{thm:final} altijd eindig is? Dat lijkt me fijn, en misschien geeft het zelfs een algoritme dat voor elke kromme een bovengrens kan geven?}
%-----------------------------------------------------------------------------------------------------------------------

\todo{First try of abstract}
Given a curve of genus at least $2$, it is known by Faltings's Theorem (1983) that the curve has only finitely many rational points. Finding the set of rational points can still be a difficult problem. Chabauty (1941) already proved the special case of Faltings's Theorem where the Mordell-Weil rank of the Jacobian is less than the genus, by intersecting the $p$-adic points of the curve with the Mordell-Weil group inside the $p$-adic points of the Jacobian, and Coleman (1985) gave a way to make this effective, by introducing Coleman integrals. This article aims to make Chabauty's method effective in a different, purely geometric way. An important part of this is working with finite precision; we will in particular focus on working modulo $p$, and will develop several heuristics, and also calculate the complexity of our algorithms.

\tableofcontents

\section{Old introduction} \todo{Put a bit of the introduction, and some of the references/results in the new introduction}
\label{section:intro}
A fundamental problem in mathematics is solving polynomial equations over the rationals, dating back to Diophantus. An important special case in algebraic geometry is that of curves. The behavior of the rational points of curves depends enormously on the \textit{genus} $g$ of the curve, a numerical invariant. For $g = 0$, there are either no or infinitely many solutions, and their behavior is well understood. For $g = 1$, we get either no point or an elliptic curve, whose set of rational points forms a finitely generated abelian group.

In this article, we look only at the case of a curve $C$ with genus $g > 1$. It turns out that there are always only finitely many rational points. This result was originally conjectured by Mordell in 1922 and finally proven by Faltings in 1983 in \cite{faltings}.

Before Faltings's theorem was proven, one of the major partial results was a theorem from Chabauty in 1941, phrased in terms of the rank $r$ of the group of rational points on the Jacobian $J$ of the curve, also called the Mordell-Weil rank; this rank is finite by the Mordell-Weil theorem. Chabauty proved, using $p$-adic methods, that if $r$ is strictly smaller than the genus $g$, then $C$ has only finitely many rational points. Loosely said, this proof and all Chabauty-related theorems, rely on choosing a prime $p$ for which $C$ has good reduction, and intersecting $J(\Q)$ with $C(\Q_p)$ inside the bigger Lie group $J(\Q_p)$. By properties of $\Q_p$, the subgroup $J(\Q)$, up to torsion generated by $r$ elements, lies within a Lie subgroup of dimension at most $r$, and $C(\Q_p)$ is a $1$-dimensional $p$-adic manifold, and hence this intersection can be proven to be discrete and -- as $J(\Q_p)$ is compact -- finite.

This was made into an effective argument by Coleman in 1985. This was done by finding explicitly, as a power series, a differential form $\omega$ on $J_{\Q_p}$ whose Coleman integral vanishes on $J(\Q)$. Pulling $\omega$ back to $C_{\Q_p}$ and looking at the fibre of reduction to a single $\F_p$-point $P$, one can in certain cases give an upper bound for the number of points in $C(\Q)$ reducing to $P$ based only on finite-precision calculations of $\omega$. Most well known is the generic upper bound $|C(\Q)| \leq |C(\F_p)| + 2g-2$ under the conditions $r < g, p > 2g$, as explained in the excellent introductory article \cite{poonen12}.

This Coleman-Chabauty method has been greatly generalised by Kim to so-called non-abelian Chabauty in \cite{kim1} and \cite{kim2}. He interprets working in the Jacobian as dealing with the abelianised fundamental group of the curve $C$ and works with larger, non-abelian quotients of the fundamental group. Then quadratic Chabauty, the simplest case of non-abelian Chabauty, was developed by Kim, Balakrishnan, Besser, Dogra and M\"uller in the series of articles \cite{qc1},\cite{qc2},\cite{qc3}, and further extended by Balakrishnan, Dogra, M\"uller, Tuitman and Vonk; in 2017 they famously calculated all rational points of the ``cursed curve'', the modular curve $X_s^+(13)$ \cite{cursedcurve}. They use an endomorphism of the Jacobian and do $p$-adic analysis on $p$-adic local heights to make quadratic Chabauty explicit. Their methods are strong enough to prove finiteness of $C(\Q)$ under the condition of $r < g + \rho - 1$, where $\rho$ is the N\'eron-Severi rank of the Jacobian, even if finding $C(\Q)$ can still be difficult in those cases.

Very recently, Edixhoven and Lido made available a preprint of their article ``Geometric quadratic Chabauty'' \cite{edixhoven20}. Their goal in this article is to create a more geometric approach to effective quadratic Chabauty, by working in a pullback $T$ of the Poincar\'e torsor of the Jacobian spread out over $\Z$. They also abandon the Coleman method for Chabauty; instead, they parametrise the map $T(\Z) \to T(\Z_p)$ with power series and pull back equations for the curve inside $T(\Z_p)$ along this map. Again, a necessary condition is $r < g + \rho - 1$.

The purpose of this article is to make the work by Edixhoven and Lido more accessible by applying geometric Chabauty to the linear case, i.e. when working with $J(\Q)$ and $J(\Q_p)$ and assuming $r < g$. In this article, we go into more detail on what is happening throughout the process, including both theoretical and practical components. This new method of applying Chabauty leads to, when working with single-digit $p$-adic precision, a conditional upper bound $|C(\Q)| \leq |C(\F_p)|$, as obtained in Proposition~\ref{prop:linchabauty}. This upper bound compares favorably with classical Coleman-Chabauty, but unlike classical Coleman-Chabauty, we cannot say in general when the conditions of these propositions hold. We develop heuristics that suggest that generally, trying multiple primes $p$ of good reduction will yield good results; these heuristics say that the conditional upper bound will hold for a set of primes of density $1$, and if $r < g-1$ the upper bound can even often (in a set of primes of density at least $e^{-1}$) be improved to prove there are at most $|C(\Q)|$ rational points, i.e., finding $C(\Q)$ exactly. The methods used to obtain Proposition~\ref{prop:linchabauty} are also amenable to higher precision calculations; indeed, in the case $r = g-1$ one expects to need either higher precision calculations or the Mordell-Weil sieve to calculate $|C(\Q)|$ exactly.

We also give a hyperelliptic example with $r = 1, g = 2$, where the bound from Proposition~\ref{prop:linchabauty} indeed holds, and we are able to compute $C(\Q)$.

\section{Introduction}\todo{Maybe give a bit more context (Faltings, Chabauty, Coleman, McCallum-Poonen, Edixhoven-Lido)? Not sure what I want to put in and what not.}\todo{Verwijzen naar dat Sachi en ik eraan werken om de connectie tussen Coleman en geometric duidelijker te maken.}
We first present the context in which we will perform Chabauty. Let $C_{\Q}/\Q$ be a curve (i.e., a proper, smooth, geometrically connected variety of dimension~$1$) of genus $g \geq 2$ whose $\Q$-points we will attempt to find. Let $p>2$ be a prime, and assume we have a scheme $C$ over $\Z_{(p)}$, proper and smooth with generic fibre $C_{\Q}$; then we immediately have $C_\Q(\Q) = C(\Q) = C(\Z_{(p)})$ by the valuative criterion of properness. Let $J$ be the relative Jacobian of $C$ over $\Z_{(p)}$, i.e. with fibres $\Jac C_\Q$ and $\Jac C_{\F_p}$. We assume from now on that the Mordell-Weil rank of $J$ is $r < g$. Assume we have a $\Q$-point $b$ in $C$, or equivalently a $\Z_{(p)}$-point. We view $C$ as a subscheme of $J$, using the map $Q \mapsto Q - b$ on points. Let $P \in C(\F_p)$ be a point such that $t := P - b \in J(\F_p)$ lies in the image of $J(\Z_{(p)})$.

\begin{defn}
Let $S$ be a scheme, $T \to U$ a morphism of schemes and $x : T \to S$ a $T$-point. We define $S(U)_x$ as the morphisms from $U$ to $S$ that, after precomposing with $T \to U$, give $x$.
\end{defn}
\begin{exmp}
If we have a proper variety $X$ over $\Z_{(p)}$, then $X(\Z_{(p)})$ is naturally in bijection with $X(\Q)$. The natural map $X(\Z_{(p)}) \to X(\F_p)$ reduces a point modulo $p$ and for $x \in X(\F_p)$, the set $X(\Z_{(p)})_x$ consists of the residue disc of $\Z_{(p)}$-points reducing to $x$.
\end{exmp}

Geometric Chabauty works by finding an upper bound for the cardinality of $C(\Z_{(p)})_P$. For this, we use the following diagram.
\[
\begin{tikzcd}
C(\Z_{(p)})_P \arrow[d] \arrow[r] & J(\Z_{(p)})_t \arrow[d] \\
C(\Z_p)_P \arrow[r]              & J(\Z_p)_t        
\end{tikzcd}
\]
where the two vertical maps are inclusions, and the two horizontal maps are subtraction of $b$. We will in fact compute upper bounds for the larger set $C(\Z_p)_P \cap \overline{J(\Z_{(p)})_t}$, where $\overline{J(\Z_{(p)})_t}$ is the closure of $J(\Z_{(p)})_t$ in $J(\Z_p)_t$. 

In goemetric Chabauty, in contrast to Coleman-Chabauty (as explained in \cite{poonen12}), we give equations for the curve inside the Jacobian, and pull those back to the Mordell-Weil group. The main result is Theorem~\ref{thm:final}, which in the case of working modulo $p$ specialises to Proposition~\ref{prop:linchabauty}. This proposition gives a conditional upper bound on $|C(\Q)|$ of at most $|C(\F_p)|$. This behaves comparably with Theorem~5.3 of \cite{poonen12}, which gives an upper bound of $|C(\F_p)| + 2g-2$, albeit under less strict conditions.

It turns out that, when working modulo $p$, the situation becomes linear algebra; we give some heuristics about how often the conditions of Proposition~\ref{prop:linchabauty} are satisfied, and what the upper bound will be in Proposition~\ref{prop:probwork} and Remark~\ref{rem:probwork}. In particular, we expect that our method will give an upper bound on $|C(\Q)|$ for all $p$ in a set of density $1$, and for $r \leq g-2$ will even lead to the optimal upper bound of $|C(\Q)|$ in a set of density at least $e^{-1}$ (with both results having refinements for even smaller $r$).

In this article, we also focus on explicit algorithms and computations. Assuming all relevant computations in the Mordell-Weil group have been done (finding the rank, a set of generators, and other data), in Section~\ref{section:explicit} we give an $O((p+g\sqrt{p})rg^{4})$ algorithm for executing the calculations needed in Proposition~\ref{prop:linchabauty}, and we implement and run this in Pari/GP.

\subsection{Overview}
In Section~\ref{section:smoothzppoints}, we treat the structure of $C(\Z_p)_P$ and $J(\Z_p)_t$; we will show that, after choosing parameters, they are in bijection with respectively $\Z_p$ and $\Z_p^g$. We also discuss the resulting map $\Z_p \to \Z_p^g$.

In Section~\ref{section:kappa} we further look at the map $J(\Z_{(p)})_t \to J(\Z_p)_t$. This is a translation of the group morphism $J(\Z_{(p)})_0 \to J(\Z_p)_0$ between kernels of reduction; it turns out that for our choice of $p$, the subgroup $J(\Z_{(p)})_0$ is free of rank $r$. We study the properties of the resulting map $\Z^r \to \Z_p^g$, using the theory of formal groups to determine the group structure on $\Z_p^g$ induced by the bijection $J(\Z_p)_0 \to \Z_p^g$.

In Section~\ref{section:intersection}, we put all of this information together, culminating in several methods to possibly compute upper bounds on $\left|C(\Z_p)_P \cap \overline{J(\Z_{(p)})_t}\right|$ with finite precision calculations. We also perform a heuristic analysis to show that a simple calculation of linear algebra modulo $p$ is very likely to result in an upper bound of at most $|C(\F_p)|$ for $|C(\Z_{(p)})|$, and for $r < g-1$ even in an upper bound of $|C(\Q)|$.

The methods developed in the first four sections are not always guaranteed to prove finiteness of $C(\Z_{(p)})_P$. In Section~\ref{section:remarks} we treat several possible complications and improvements, referring to other work on Chabauty for further reading.

Next, in Section~\ref{section:explicit} we focus on how to perform the calculations happening in the Jacobian. Here, we use Makdisi's approach of representing a divisor by a subspace of a large Riemann-Roch space. We partially follow the article \cite{mascot18} by Mascot, and also give an algorithm for explicitly computing the map $C(\Z_p)_P \to J(\Z_p)_t$ in Makdisi's representation.

We end our discussion of linear geometric Chabauty with an explicit example in Section~\ref{section:example}, a genus $2$ curve whose Jacobian has Mordell-Weil rank $1$. We use both Magma and Pari/GP to do our calculations, and end up with a complete list of all rational points of the curve.

%%%%%%%%%%%%%%%%%%%%%%%%%%%%%%%%%%%%%%%%%%%%%%%%%%%%%%%%%%%%%%%%%%%%%%%%%%%%%%%%%%%%%%%%%%%%%%%%%%%%%%%%%%%%%%%%%%%%%%%%%%%%%%%%%%%%%%%%%%%%%%%%%%%%%%%%%%%%
\newpage
% Say something about smooth schemes over Z_p, and their Z_p points
\section{Points on a smooth scheme over \texorpdfstring{$\Z_p$}{Z\_p}}
\label{section:smoothzppoints}
Let $X/\Z_p$ be a smooth scheme of relative dimension $d$, and let $x \in X(\F_p)$ be a point. Then we will show in this section that $X(\Z_p)_x$ is, up to a single choice, naturally in bijection with $\Z_p^d$. This bijection is given by choosing parameters at $x$; evaluating at $X(\Z_p)_x$ gives a bijection with $(p\Z_p)^d$, and then we divide by $p$. For putting up a nice framework to work in, we start with blowing up $X$ at $x$.

% Blow up X at x, and look at the part where p generates the maximal ideal
Assume, by looking at a neighborhood of $x$, that $X = \Spec A$ is affine and that $p,t_1,\dots,t_d$ generate the maximal ideal of $\O_{X,x}$ with $t_1,\dots,t_d$ elements of $\O_X(X) = A$, also called \textit{parameters} at $x$. By shrinking $X$ even more, we may assume as $X$ is smooth that $t = (t_1,\dots,t_d): X \to \A^d_{\Z_p}$ is \'etale and the fibre of the origin over $\F_p$ consists of just $x$. Now consider the blowup $\widetilde{X}_x \to X$ of $X$ at $x$, and let $\widetilde{X}_x^p$ be the open subscheme where $p$ generates the inverse image of the maximal ideal of $\O_{X,x}$. Equivalently, that is the part where $t_1,\dots,t_d$ are multiples of $p$, so informally $\widetilde{X}_x^p$ consists of the points that reduce to $x$ modulo $p$.

There is an explicit description of the map $\widetilde{X}_x^p \to X$; as $t$ is \'etale, the ideal of $X$ defining $x$ is the pullback along $t$ of the ideal of $\A^d_{\Z_p}$ defining the origin $a$ over $\F_p$. That means that the blowup $\widetilde{X}_x \to X$ is the pullback of the blowup $\widetilde{\A}^d_{\Z_p,a} \to \A_{\Z_p}^d$. Then the open subscheme $\widetilde{X}_x^p$ is the pullback of the corresponding subscheme of $\widetilde{\A}^d_{\Z_p,a}$, i.e. $\Spec \Z_p[\widetilde{x_1},\dots,\widetilde{x_d}] = \Spec \Z_p[x_1/p,\dots,x_d/p]$ with the morphism $\Z_p[x_1,\dots,x_d] \to \Z_p[\widetilde{x_1},\dots,\widetilde{x_d}]$ given by $x_i \mapsto p\widetilde{x_i}$. That implies that $\widetilde{X}_x^p$ is $\Spec A[t_1/p,\dots,t_d/p]$, with the map $\widetilde{X}_x^p \to X$ given by the inclusion $A \to \Spec A[t_1/p,\dots,t_d/p]$ (remember that the $t_i$ are elements of $\O_X(X) = A$).

This now enables us to characterise explicitly the $\Z_p$-points above $x$, as in the following two lemmas.

\begin{lem}
There is a natural bijection \[X(\Z_p)_x \to \widetilde{X}_x^p(\Z_p).\]
\end{lem}
\begin{proof}
Note that by (I,2.4.4) of \cite{ega}, a $\Z_p$-point of a scheme $S$ is just an $\F_p$-point $s$ together with a local morphism $\O_{S,s} \to \Z_p$. In our case, we find that $X(\Z_p)_x$ is naturally in bijection with $\Hom_{\text{local}}(A_x,\Z_p)$. As the maximal ideal of $A_x$ is generated by $p,t_1,\dots,t_d$, the morphism being local just means that the images of $t_1,\dots,t_d$ are divisible by $p$. That exactly gives those morphisms that extend to a morphism $A[t_1/p,\dots,t_d/p] \to \Z_p$, i.e. a $\Z_p$-point of $\widetilde{X}_x^p$. Hence we find the natural bijection.
\end{proof}

\begin{lem}
Evaluating $t$ at $X(\Z_p)_x$ gives a bijection to $(p\Z_p)^d$.
\end{lem}
\begin{proof}
As $t$ is locally of finite type, by (IV,17.6.3) of \cite{ega} we have an isomorphism between the $p$-adic completion $\O(\widetilde{X}_x^p)^{\wedge p}$ and the completion $\Z_p\angle{\widetilde{x_1},\dots, \widetilde{x_d}}$ of $\Z_p[\widetilde{x_1},\dots,\widetilde{x_d}]$, with the latter completion being the ring of convergent power series, i.e.
\[
\Z_p\angle{\widetilde{x_1},\dots, \widetilde{x_d}} = \left\{f \in \Z_p[[\widetilde{x_1},\dots,\widetilde{x_d}]] \mid \forall n \geq 0, f \in \Z[\widetilde{x_1},\dots,\widetilde{x_d}] + (p^n) \right\}.
\]
By the universal property of completions, as $t$ induces the isomorphism between completions, $t$ also induces a bijection $$\Hom(\O(\widetilde{X}_x^p), \Z_p) \to \Hom(\Z_p\angle{\widetilde{x_1},\dots,\widetilde{x_d}},\Z_p) = \Z_p^d.$$ Following all the bijections, we indeed get the bijection we wanted.
\end{proof}

\begin{rem}
\label{rem:smoothpointsfunc}
Note that this construction is functorial, in the following sense: let $X,Y$ be two smooth schemes over $\Z_p$ and $x\in X(\F_p)$, $y\in Y(\F_p)$ be two $\F_p$-points, and let $f: Y \to X$ be a map satisfying $f(y) = x$. Then the inverse image along the map\[\widetilde{Y}_y^p \to Y \to X\] of the ideal sheaf defining $x$ is the ideal generated by $p$. Hence by the universal property of blowups (II.7.14 of \cite{hartshorne}), this factors through a unique morphism $\widetilde{Y}_y^p \to \widetilde{X}_x$, landing in $\widetilde{X}_x^p$, i.e. we get that there is a unique morphism $\widetilde{f}: \widetilde{Y}_y^p \to \widetilde{X}_x^p$ making the diagram
\[
\begin{tikzcd}
\widetilde{Y}_y^p \arrow[d] \arrow[r,"\widetilde{f}"] & \widetilde{X}_x^p \arrow[d] \\
Y \arrow[r,"f"]              & X  
\end{tikzcd}
\]
commute.
\end{rem}

\begin{rem}
\label{rem:torsortangentspace}
In actual calculations, we will be focusing on $X(\Z/p^2\Z)_x$. This set is a natural torsor of the tangent space $T_x X_{\F_p}$ of $X_{\F_p}$ at $x$, which we will describe here. We write $R$ for the local ring $\O_{X_{\Z/p^2\Z},x}$ and $\m$ for its maximal ideal, and $\overline{R}$ and $\overline{\m}$ for their reductions modulo $p$. Then an element in $X(\Z/p^2\Z)_x$ is a local morphism $R \to \Z/p^2\Z$. As $R/\m = \F_p$, giving such a local morphism is equivalent to giving a map $\m \to p\Z/p^2\Z$ respecting multiplication and sending $p$ to $p$. Such a map factors uniquely through $\m/\m^2 \to p\Z/p^2\Z$. We see that the set $X(\Z/p^2\Z)_x$ is canonically in bijection with $\F_p$-linear maps $\m/\m^2 \to p\Z/p^2\Z$ sending $p$ to $p$. Note that $\m/\m^2$ is a $(d+1)$-dimensional $\F_p$-vector space, and $\overline{\m}/\overline{\m}^2$ is a $d$-dimensional $\F_p$-vector space; the map $\m/\m^2 \to \overline{\m}/\overline{\m}^2$ is dividing out by $p$. Denoting ${}^\wedge$ to be $\Hom(\cdot,\F_p)$, the exact sequence
\[
0 \to p\Z/p^2\Z \to \m/\m^2 \to \overline{\m}/\overline{\m}^2 \to 0
\]
can be dualised to
\[
0 \to (\overline{\m}/\overline{\m}^2)^\wedge \to (\m/\m^2)^\wedge \to (p\Z/p^2\Z)^{\wedge} \to 0.
\]
As $X(\Z/p^2\Z)_x$ is exactly the subset of $(\m/\m^2)^\wedge$ that gets mapped to the function $p \mapsto 1 \in (p\Z/p^2\Z)^{\wedge}$, we see this is naturally a torsor under the tangent space $(\overline{\m}/\overline{\m}^2)^\wedge = T_{x} X_{\F_p}$.

Note that this set does not have any more canonical structure; for example, with $X = \A^1_{\Z_p}$ and $x$ the $\F_p$-point $1$, the set $X(\Z/p^2\Z)_x$ is $\{1 + pi | i \in \F_p\}$, and we see we cannot upgrade the torsor structure. This is because even after choosing a parameter at $x$ as a point of $X_{\F_p}$, a lift of such a parameter to $X_{\Z/p^2\Z}$ is not canonical, and can in fact differ up to translation. This is in contrast to the situation over $\F_p[\varepsilon]/(\varepsilon^2)$, where the $\F_p$-algebra structure gives rise to an isomorphism of $\F_p$-vector spaces between $X(\F_p[\varepsilon]/(\varepsilon^2))$ and $T_x X_{\F_p}$; indeed, this is an alternate definition of the tangent space.
\end{rem}

\begin{rem}
\label{rem:cartierlinear}
We look at the specific case of $X$ being of relative dimension~$1$ over $\Z_p$. Then the $T_x X_{\F_p}$-torsor structure on $X(\Z/p^2\Z)_x$ translates into something more concrete. We can parametrise $X(\Z/p^2\Z)_x$ by $t = t_1 $ from $X(\Z/p^2\Z)_x \to p\Z/p^2\Z$; write $P_{\lambda}$ for the point with $t$-value $\lambda p$. Then as a Cartier divisor, the point $P_{\lambda}$ is defined by $t - \lambda p \in \O_{X_{\Z/p^2\Z},x}$, so $P_{\lambda} + P_{\mu}$ is defined by $(t-\lambda p)(t-\mu p) = t^2 - (\lambda + \mu)tp$, which defines the same Cartier divisor as $P_{\lambda'} + P_{\mu'}$ if and only if $\lambda + \mu = \lambda' + \mu'$. In that case, as a Cartier divisor, $P_{\lambda} + P_{\mu}$ is in fact equal to $P_{\lambda'} + P_{\mu'}$.
\end{rem}

\subsection{From \texorpdfstring{$C(\Z_p)$}{C(Z\_p)} to \texorpdfstring{$J(\Z_p)$}{J(Z\_p)}}
\label{subsection:czptojzp}
Let $p$, $C$ and $J$ be as defined in the overview. We know that $C(\Z_p)_P$ is in bijection with $\Z_p$, again with the bijection given by evaluating a parameter and dividing by $p$. The resulting function $\Z_p \to \Z_p^g$ is linear modulo $p$, i.e. by using that $\Z\angle{z_1,\dots,z_g}$ is $p$-adically complete there are power series $f_1,\dots,f_{g-1} \in \Z_p\angle{z_1,\dots,z_g}$ such that the image of $C(\Z_p)_P$ is exactly given by $Z(f_1,\dots,f_{g-1})$, and all $f_i$ are linear modulo $p$ as a consequence of Remark \ref{rem:cartierlinear}. Another way to think of this, is as $C(\Z/p^2\Z)_P$ being an affine line inside $J(\Z/p^2\Z)_t$. That $C(\Z/p^2\Z)_P$ is an affine line inside $J(\Z/p^2\Z)_t$ can also be seen more easily: this map can be identified with the tangent map, using the structure of $X(\Z/p^2\Z)_x$ as a torsor over the tangent space from Remark~\ref{rem:cartierlinear}.

\begin{rem}
\label{rem:fislinear}
We can pick our parameters such that $C(\Z_p)_P \to J(\Z_p)_t$ is given by $\Z_p \to \Z_p^g$, mapping $\Z_p$ to the last coordinate. Then $f_1,\dots,f_{g-1}$ are just the other parameters at $t$, so they are linear. This can be done by picking generators $f_1,\dots,f_{g-1}$ for the ideal defining $\widetilde{C}_P^p$ inside $\widetilde{J}_t^p$.
\end{rem}

\section{From \texorpdfstring{$J(\Z_{(p)})$}{J(Z\_{(p)})} to \texorpdfstring{$J(\Z_p)$}{J(Z\_p)}}
\label{section:kappa}\todo{Try to find a reference for this section. Maybe try "Lie groups and Lie algebras" by Serre. It seems to almost be folklore, couldn't find a reference by googling. And I can't even find my result on exp and log mod powers of p.}
Let $p,C,J$ be as in the overview. As $p > 2$, we know that the torsion of $J(\Z_{(p)})$ injects into $J(\F_p)$, by Proposition 2.3 of \cite{pierre2000}. Hence for $0 \in J(\F_p)$, we know $J(\Z_{(p)})_0$ is as a group isomorphic to $\Z^r$ with $r$ the Mordell-Weil rank. By assumption, we also know $J(\Z_{(p)})_t$ is in bijection with $J(\Z_{(p)})_0$, with the bijection giving by translating with a lift of $t$. By Section \ref{section:smoothzppoints} we know $J(\Z_p)_t$ is in bijection with $\Z_p^{g}$, with the bijection given by evaluating parameters and dividing by $p$. Let $\kappa: \Z^r \to \Z_p^g$ be the map resulting from the inclusion $J(\Z_{(p)})_t \to J(\Z_p)_t$. In this section we will prove that $\kappa$ turns out to have a special property.

\begin{thm}
\label{thm:kappanice}
There are uniquely determined $\kappa_1,\dots,\kappa_g \in \Z_p\angle{z_1,\dots,z_r}$ such that for all $\vx \in \Z^r$ we have $\kappa(\vx) = (\kappa_1(\vx),\dots,\kappa_g(\vx))$ and the image $\overline{\kappa_i}$ of $\kappa_i$ in $\F_p[z_1,\dots,z_r]$ has degree at most $1$; furthermore, for $m \in \Z_{>0}$ with $m<p-1$, these $\kappa_i$ are also of degree at most $m$ modulo $p^m$.
\end{thm}

We will prove this using results about formal groups as defined in \cite{honda70}. To be able to use this theory, we first give some results about going from a group scheme over $\Z_p$ to a formal group.
We introduce some notation that will be used in this section. Let $R$ be a commutative ring, and let $\vx = (x_1,\dots,x_n)$ and $\vy = (y_1,\dots,y_m)$ be vectors of variables. Then $R[[\vx]]$ denotes as usual the ring of formal power series in the $x_i$, and $R[[\vx]]_0$ denotes those power series with constant term $0$. With $\vx = (x_1,\dots,x_n)$ and $\vy = (y_1,\dots,y_m$), let $f \in R[[\vx]]_0^m$. Then for $g \in R[[\vy]]^k$ for some $k\in\Z_{\geq 0}$, we can compose $g$ and $f$ to get \[g\circ f := (g_1(f(\vx)),\dots,g_k(f(\vx)))\in R[[\vx]]^k.\] This definition makes sense because $f^i$ converges to $0$ in the $(\vx)$-adic topology, so the infinite sum $g_j(f(\vx))$ converges.

\subsection{From group schemes to formal groups}
We first recall the definition of a formal group.
\begin{defn}
\label{defn:formalgroup}
Let $n$ be a non-negative integer. Let $\vx,\vy,\vz$ be vectors of $n$ variables. An $n$-dimensional formal group is an element $F = (F_1,\dots,F_n)$ of $R[[\vx,\vy]]_0^n$ with $F \equiv \vx + \vy \bmod (\vx,\vy)^2$ and $F(F(\vx,\vy),\vz) = F(\vx,F(\vy,\vz))$. If furthermore $F(\vx,\vy) = F(\vy,\vx)$, this formal group is said to be commutative.
\end{defn}
\begin{exmp}\todo{If I can't find a reference, at least remove this examples.}
\label{exmp:ganformalgroup}
For any $n$, we can take $F(\vx,\vy) = \vx+ \vy$, also known as the $n$-dimensional additive formal group.
\end{exmp}
\begin{exmp}
\label{exmp:gmformalgroup}
Take $n=1$ and $F(x,y) = x + y + xy = (1+x)(1+y)-1$, also known as the multiplicative formal group, as it is a translation of the natural multiplication on $1 + xR[x]$. This satisfies associativity by the formula \[F(F(x,y),z) = (1+x)(1+y)(1+z)-1 = F(x,F(y,z)).\]
\end{exmp}
Note there is no mention of an inverse; the following lemma, a formal version of the implicit function theorem, tells us that the existence and uniqueness of the inverse follows automatically from the definitions.
\begin{lem}
\label{lem:implicitfunction}
Let $\vx,\vy$ be vectors of variables of length $n$. Let $F \in R[[\vx,\vy]]_0^n$ such that $F \equiv A\vx + B\vy \bmod (\vx,\vy)^2$ with $A \in \Mat_n(R)$ and $B \in \GL_n(R)$. Then there is a unique $\iota \in R[[\vx]]_0^n$ such that $F(\vx,\iota(\vx)) = 0$. 
\end{lem}
\begin{proof}
Note that $R[[\vx]]$ is complete with respect to the ideal $\m = (\vx)$, and the derivative matrix of $F(\vx,\iota)$ with respect to $\iota$ is $B$, which is invertible, and $\iota = 0$ gives a solution modulo $\m$. If $F$ is a polynomial, this means the existence and uniqueness of $\iota$ follow directly from the multivariate version of Hensel's lemma (Corollaire~2 of \cite[III,4.5]{BourbakiCA}).  Now for general $F \in R[[\vx,\vy]]_0^n$, let $F_j \in R[\vx,\vy]_0^n$ consist of all terms in $F$ of degree at most $j$, and let $\iota_j$ be the unique power series in $R[[\vx]]_0^n$ such that $F_j(\vx,\iota_j) = 0$. Note that both $\iota_j$ and $\iota_{j+1}$ are solutions to $F_j(\vx,\iota) \equiv 0 \bmod \m^j$, so by the uniquess guaranteed by Hensel's lemma used over $R[[\vx]]/\m^j$, these must be equal. Hence they converge to $\iota \in R[[\vx]]_0^n$, which is the unique solution of $F(\vx,\iota(\vx)) = 0$.
\end{proof}
This has the following corollary about the inverse of a power series.
\begin{cor}
\label{lem:formalinverse}
Let $\vx$ have length $n$. Let $a \in R[[\vx]]_0^n$ with $a \equiv P\vx \bmod (x)^2$ for some matrix $P\in \GL_n(R)$. Then there is a unique $b \in R[[\vx]]_0^n$ such that $a \circ b = b \circ a = x$.
\end{cor}
\begin{proof}
Let $F(\vx,\vy) = \vx - a(\vy)$. This satisfies the conditions of Lemma~\ref{lem:implicitfunction}, so we find a unique $b$ such that $a \circ b = \vx$. Applying Lemma~\ref{lem:implicitfunction} again gives a unique $c$ such that $b \circ c = \vx$. But then $a = a \circ (b \circ c) = (a \circ b) \circ c = c$ shows $a = c$ and hence we are done.
\end{proof}
So a formal group $F$ does have a right inverse $\iota_F$. Also, by $F(0,0) = 0$ we have that $F(\vx,F(\iota_F,0)) = 0$ so $F(\iota_F,0)$ is in fact equal to $\iota_F$; as $\iota_F \equiv -\vx \bmod (\vx)^2$, it has a formal inverse and hence $F(\vx,0) = \vx$, so $0$ is indeed a right unit. A standard argument now shows that $\iota_F$ is also a left inverse and $0$ is also a right unit, so $F$ is indeed a group law in the following sense. We consider the category of pairs of objects $(A,I)$ where $A$ is a topological $R$-algebra, complete with respect to $I$ with the $I$-adic topology. We give $(R[[\vx]],(\vx))$, together with the $(\vx)$-adic topology, the structure of a group object in this category for two continuous morphisms $f_1,f_2$ from $(R[[\vx]],(\vx))$ to $(A,I)$, given by sending $\vx$ to $\va_1,\va_2 \in I^n$ respectively, we define $(f_1 \oplus f_2)(x) = F(\va_1,\va_2)$. By our considerations, this makes $\Hom((R[[\vx]],(\vx)),(A,I))$ into a group, functorially in $(A,I)$.

Given a smooth scheme $G$ of relative dimension $d$ over a ring $R$, together with a $R$-point $e$, we can look at the completion $\widehat{\O}_{G,e}$ along the section $e$. By smoothness, after picking parameters, this is isomorphic as a topological $R$-algebra to the ring of power series $R[[x_1,\dots,x_d]]$. This completion naturally gives rise to a formal scheme denoted $\Spf \widehat{\O}_{G,e}$; in a categorical view, this is a functor on finite length $R$-algebras sending $A$ to $\Hom_{\text{cont}}(\widehat{\O}_{G,e},A)$, but we can also think of it as a locally ringed space with $\Spec R$ as topological space, and sheaf of rings the inverse limits of the sheaf of rings associated to the $R$-algebra $\O_{G,e}/I^n$, where $I$ is the ideal of $\O_{G,e}$ defining $e$. The global sections of this sheaf are simply $\widehat{\O}_{G,e}$. 

If furthermore $G$ is a group scheme over $R$ and $e$ is the unit section, then this formal scheme promotes to a formal group scheme, i.e. for every finite $R$-algebra $A$ the set $\Hom_{\text{cont}}(\widehat{\O}_{G,e},A)$ gets a group structure, functorially in $A$. By functoriality, this is the same as an continuous coproduct \[\mu: \widehat{\O}_{G,e} \to \left(\widehat{\O}_{G,e}\right)^{\widehat{\tensor} 2}.\]
After choosing an isomorphism between $\widehat{\O}_{G,e}$ and $R[[x_1,\dots,x_d]]$, let the power series $F_j \in R[[\vx,\vy]]$ denote $\mu(x_j)$. Then it is clear that \[F_G = (F_1,\dots,F_d)\] is a $d$-dimensional formal group scheme, and is commutative if $G$ is commutative.
\addtocounter{thm}{-3}
\begin{exmp}[continued]
\label{exmp:gmtoformal}
Take $G = \G_m = \Spec \Z_p[u,u^{-1}]$ over $\Z_p$. Then the zero section $e$ is the map sending $u$ to $1$, and we can identify the completion $\widehat{\O}_{G,e}$ with the power series ring $\Z_p[[u-1]] \cong \Z_p[[x]]$, sending $u-1$ to $x$. The group structure on $G$ then gives rise to the coproduct $\Z_p[[u-1]] \to \Z_p[[u-1,v-1]], u \mapsto uv$ so the coproduct on $\Z_p[[x]]$ sends $x$ to $uv-1 = (x+1)(y+1) -1 = x + y + xy$. Hence the formal group $F_{\G_m}$ corresponding to this group scheme is exactly the multiplicative formal group from Example~\ref{exmp:gmformalgroup}.
\end{exmp}

Note that if $R = \Z_p$, the formal group $F_G$ tells us exactly how multiplication works on $G(\Z/p^e\Z)_0$; if we let $t = (t_1,\dots,t_d)$ denote the formal parameters giving the isomorphism $\widehat{\O}_{G,e} \to \Z_p[[x_1,\dots,x_d]]$, and we represent a point $P$ in $G(\Z/p^e\Z)_0$ by their parameter values $t(P) \in p\Z/p^e\Z$, then the resulting multiplication on $p\Z/p^e\Z$ obtained from the group structure on $G(\Z/p^e\Z)_0$, i.e. the map $p\Z/p^e\Z \times p\Z/p^e\Z \to p\Z/p^e\Z$, is exactly evaluating $F_G$. By taking limits, $F_G$ also gives the multiplication on $G(\Z_p)_0$.

Now we will look at some standard results on formal groups. For an $n$-dimensional formal group $F$ over a ring $R$, Proposition~1.1 of \cite{honda70} gives us a canonical $R$-basis $\omega_1,\dots,\omega_n$ of the right invariant differentials; these are elements of $\bigoplus_{j=1}^n R[[\vx]] \diff x_j$. If the formal group is furthermore commutative, then by Proposition~1.3 of \cite{honda70} these are closed, i.e. $d\omega_j = 0$. A $1$-form $\omega = \sum_{i=1}^n f_i \diff x_i$ being closed means exactly that $\frac{\partial f_i}{\partial x_j} = \frac{\partial f_j}{\partial x_i}$ for all $i,j$. From now on, assume $R$ has no torsion; then $R$ embeds into $\Q \tensor R$ and, we can formally integrate $\omega$, i.e. write it as $df$ where $f \in (\Q\tensor R)[[\vx]]$. We make this unique by demanding that $f(0) = 0$. Writing this $f$ as $\sum_{I \in \N^n} a_I x^I$ with $I$ denoting a multi-index and $a_I \in \Q\tensor R$, we see that $a_I$, although itself not necessarily lying in $R$, is close: writing $I = (I_1,\dots,I_n)$ we see that for all $j$ we must have $I_j a_I \in R$ as $f_j = \frac{\partial f}{\partial x_j} \in R[[\vx]]$.

In particular, we write $\log_i$ for the unique element of $(\Q\tensor R)[[\vx]]_0$ such that $d\log_i = \omega_i$. Together, these $\log_i$ give an element $\log \in (\Q\tensor R)[[\vx]]_0^n$, and by Theorem~1 of \cite{honda70}, this logarithm satisfies $\log(\vx) = \vx \bmod (\vx)^2$, and $\log(F(\vx,\vy)) = \log(\vx)+\log(\vy)$. By the first result, $\log$ has an inverse which we will call $\exp$, also given by power series in $(\Q\tensor R)[[\vx]]_0^n$, and also satisfying $\exp(\vx) = \vx \bmod (\vx)^2$.
\addtocounter{thm}{-1}
\begin{exmp}[continued]
Taking $F$ again the multiplicative formal group as in Example~\ref{exmp:gmformalgroup}, Proposition~1.1 of \cite{honda70} gives us $\omega = \frac{1}{x+1}dx$; as a power series this is $\sum_{j \geq 0} (-x)^jdx$. The formal anti-derivative of this is \[\log = \sum_{j \geq 1} \frac{-(-x)^j}{j}\] (note this is a translate of the $\log$ from analysis), and Theorem~$1$ of \cite{honda70} will tell us what we already know in the context of analysis, namely the formula $\log(x + y + xy) = \log(x)+\log(y)$ (this formula is also a translate of the corresponding formula from analysis). Then analysis will also explicitly tell us what $\exp$ looks like: $\exp(x) = \sum_{j \geq 1} \frac{x^j}{j!}$.
\end{exmp}
\addtocounter{thm}{2}

Now we will see how we can use this formal logarithm in our case of the formal group $F_G$ stemming from a $d$-dimensional smooth group scheme $G/\Z_p$. Recall from Section~\ref{section:smoothzppoints} that we have a bijection $G(\Z_p)_0 \to p\Z_p^d \to \Z_p^d$ given by evaluating at parameters $t = (t_1,\dots,t_d)$ and then dividing by $p$; furthermore, recall that the group structure on $G(\Z_p)_0$ is the same as the group structure defined by $F_G$ on $p\Z_p^d$. Let $\log \in \Z_p[[\vx]]^d$ denote the logarithm corresponding to $F_G$, and let $\log_p$ denote the power series \[\log(p\vx)/p = (\log_1(p\vx)/p,\dots,\log_d(p\vx)/p).\]
We then have the following little lemma to bridge the gap between power series and maps.
\begin{lem}
\label{lem:formallog}
With notation as above, we have the following statements about $\log_p$, the first two of which also hold for $p=2$:
\begin{enumerate}
	\item $\log_p$ lies in $\Z_p\angle{x_1,\dots,x_d}^d$, the ring of convergent power series, and hence defines a map $\log_p: \Z_p^d \to \Z_p^d$.
	\item Letting $\oplus$ denote the group structure on $\Z_p^d$ coming from the bijection $G(\Z_p)_0 \to p\Z_p^d \to \Z_p^d$, the map $\log_p$ is a group morphism from $(\Z_p^d,\oplus)$ to $(\Z_p^d,+)$.
	\item If $p > 2$, the map $\log_{p,i}$ reduces to $x_i$ modulo $p$ and hence $\log_p$ has an inverse $\exp_p$, which is also an element of $\Z_p\angle{x_1,\dots,x_d}_0^d$. Then this $\exp_p$ is a two-sided inverse of $\log_p$, and hence $\log_p: \Z_p^d \to \Z_p^d$ is a bijection.
	\item Let $m \in \Z_{>0}$. For $p > m+1$, the maps $\log_p$ and $\exp_p$ are of degree at most $m$ modulo $p^m$.
\end{enumerate} 
\end{lem}
\begin{proof}
Write $\log_{i} = \sum_{I} a_{i,I} \vx^I$ and $\log_{p,i} = \sum_I a_{i,I} p^{|I|-1} \vx^I$ where $I$ ranges over the multi indices, and $|I| = \sum_{i=1}^d I_i$. Recall that $a_{i,0} = 0$. Also, recall that although a priori the coefficients $a_I$ lie in $\Q_p$, we have $I_j a_{i,I} \in \Z_p$ for all $I$ and $i,j$. For purposes of easy estimation, this also means $|I|a_{i,I} \in \Z_p$. As we have, with no constraint on $p$, for all $n\geq 1$ that $c_n\coloneqq p^{n-1}/n$ lies in $\Z_p$ and converges to $0$, this means that \[ \log_{p,i} = \sum_I a_{i,I} p^{|I|-1} \vx^I = \sum_I |I|a_{i,I} c_{|I|} \vx^I\] is indeed an element of $\Z_p\angle{x_1,\dots,x_d}^n$, and hence defines a map $\Z_p^d \to \Z_p$. Hence all the $\log_{p,i}$ together give a map $\Z_p^d \to \Z_p^d$. By the equality of power series $\log(F(\vx,\vy)) = \log(\vx)+\log(\vy)$ from the definition of the logarithm, and the fact that all these power series converge on $p\Z_p^d$, this $\log_{p,i}$ is indeed a group morphism from $(\Z_p^d,\oplus)$ to $(\Z_p^d,+)$.

To study $\log_p$ modulo $p$, we note that if $n\geq 3$, then $p$ actually divides $c_n$, and for $p > 2$ we even have $p|c_2$. As $\log_{p,i} = x_i \bmod (\vx)^2$, this means that for $p > 2$ we have \[\log_{p,i} \equiv \sum_{|I| = 1} |I|a_{i,I}c_{|I|} x^I \equiv x_i \bmod p.\]
Note that as $\log_p \equiv \vx \bmod (\vx)^2$, by Lemma~\ref{lem:formalinverse} it has an inverse $\exp_p$ that a priori lies just in $\Z_p[[\vx]]_0^n$. Modulo $p$, this $\exp_p$ must be $x$, which lies in $\Z_p\angle{\vx}_0^n$; as $\Z_p\angle{\vx}$ is $p$-adically complete, using Hensel's lemma this shows $\exp_p$ is indeed in $\Z_p\angle{\vx}_0^n$.

Finally, let $m \in \Z_{>0}$ with $p > m+1$. As $c_n$ has $p$-adic valuation $n-1$ for $n < p$, and $p$-adic valuation at least $p-2 \geq m$ for $n \geq p$, indeed $\log_p$ has degree at most $m$ modulo $p^m$. That the same holds for $\exp_p$, follows immediately from the consideration that the set
\[
S \coloneqq \{f \in \Z/p^n\Z[\vx]_0^d \mid  f \equiv \vx \bmod p, \forall m \deg(f \bmod p^m) \leq m\}
\]
forms a group under composition, as we will now see. As $\Z_p\angle{\vx}$ is complete, we know the superset of $S$
\[
T \coloneqq \{f \in \Z/p^n\Z[\vx]_0^d \mid  f \equiv \vx \bmod p\}
\]
is a group. As $S$ is finite and non-empty, all that remains to show is that $S$ is closed under composition. We can also characterise $S$ as consisting of images $\overline{f}$ of polynomials $f \in \Z_p[\vx]_0^d$ such that there is a polynomial $\widetilde{f}\in\Z_p[\vx]_0^d$ with $pf(\vx) = \widetilde{f}(p\vx)$. If we then take two elements $\overline{f},\overline{g} \in S$, reductions of $f,g$ respectively with $\widetilde{f},\widetilde{g}$ such that $pf(\vx) = \widetilde{f}(p\vx), pg(\vx) = \widetilde{g}(p\vx)$, we see that with $h = f\circ g, \widetilde{h} = \widetilde{f} \circ \widetilde{g}$ that $ph(\vx) = \widetilde{h}(p\vx)$ and $\overline{h} \equiv \overline{f} \circ \overline{g} \bmod p^m$, so indeed $\overline{f} \circ \overline{g}$ is also an element of $S$. Hence we see $S$ is indeed a group, and as $\log_p \bmod p^m$ is an element of it, so is $\exp_p \bmod p^m$.
\end{proof}

This lemma makes the proof of Theorem~\ref{thm:kappanice} relatively easy. 
\begin{proof}[Proof of Theorem~\ref{thm:kappanice}]
Write $\log$ for the logarithm corresponding to the formal group of $J$ and $\log_p$ for $\log(p\vx)/p$ and $\exp_p$ for its inverse. Note that the map $\kappa_0: \Z^d \to \Z_p^g$ given by the inclusion $J(\Z_{(p)})_0$ to $J(\Z_p)_0$ is a group morphism with the group structure on $\Z^d$ being addition and the group structure on $\Z_p^g$, denoted by $\oplus$, given by the bijection with $J(\Z_p)_0$ induced by choosing parameters at $0$, and $\kappa$ is $\kappa_0 \oplus t$. Then by Lemma~\ref{lem:formallog} the map $\kappa$ factors as in the diagram of groups
\[
% https://tikzcd.yichuanshen.de/#N4Igdg9gJgpgziAXAbVABwnAlgFyxMJZABgBpiBdUkANwEMAbAVxiRAAoAdTgLQD0opANQBKEAF9S6TLnyEUAJnJVajFmy68A+mj4BzUtwhpmcMZOnY8BIgEZStlfWatEHbjx37h5lTCh68ESgAGYAThAAtkhkIDgQSPYgDHQARjAMAAoy1vLJMCE4INTO6m7cANZ0aGh0ElIg4VEx1PFISqoubNwMEHo6AATcAMZYYcNDnFU1dRaNEdGIHW2ISaWuINwwAB5oOhIU4kA
\begin{tikzcd}
{(\Z^d,+)} \arrow[rr, "\kappa"] \arrow[rd, "\log_p \circ \kappa"] &                                   & {(\Z_p^g,\oplus)} \\
                                                                  & {(\Z_p^g,+)} \arrow[ru, "\exp_p"] &                  
\end{tikzcd}
\]
Then the arrow $\log_p \circ \kappa$ is just a linear map, and $\exp_p$ is a convergent power series that is of degree at most $m$ modulo $p^m$ for $m < p-1$, so their composite $\kappa$ is also a convergent power series of degree at most $m$ modulo $p^m$ for $m < p-1$.
\end{proof}

This immediately gives rise to the following corollary.
\begin{cor}
\label{cor:closurejac}
The map $\kappa: \Z^r \to \Z_p^g$ extends uniquely to a continuous map $\kappa: \Z_p^r \to \Z_p^g$, given by the same power series, and the closure $\overline{J(\Z_{(p)})_t} \subset J(\Z_p)_t$ is given by the image of $\Z_p^r$ under $\kappa$.
\end{cor}

\section{Computing the intersection}
\label{section:intersection}
Let $p,C$ and $J$ be as defined in the overview. Clearly, as per the diagram \[
\begin{tikzcd}
C(\Z_{(p)})_P \arrow[d] \arrow[r] & J(\Z_{(p)})_t \arrow[d] \\
C(\Z_p)_P \arrow[r]              & J(\Z_p)_t        
\end{tikzcd}
\]
we have, as subsets of $J(\Z_p)_t$, the inclusion \[C(\Z_{(p)})_P \subset C(\Z_p)_P \cap \overline{J(\Z_{(p)})_t}.\]
The theory we have built so far enables the following method, which is in sharp contrast with Coleman's method; instead of pulling back equations for $\overline{J(\Z_{(p)})_t}$ to $C(\Z_p)_P$, we pull back equations for $C(\Z_p)_P$ to $\overline{J(\Z_{(p)})_t}$ to arrive at the following theorem.
\begin{thm}
\label{thm:final}
Let $C,J,p,P\in C(\F_p),t\in J(\F_p)$ be as in the overview, let $\kappa: \Z_p^r \to \Z_p^g$ be as in Theorem~\ref{thm:kappanice} and let $f_1,\cdots,f_{g-1} \in \Z_p\angle{x_1,\dots,x_g}$ be as in Subsection~\ref{subsection:czptojzp}. Define \[\lambda_1 = \kappa^*f_1,\dots,\lambda_{g-1} = \kappa^*f_{g-1}\] to be the pullbacks along $\kappa$ of the $f_i$. Then $\kappa$ induces a surjection from $Z(\lambda_1,\dots,\lambda_{g-1}) \subset \Z_p^r$ to $C(\Z_p)_P \cap \overline{J(\Z_{(p)})_t}$.
\end{thm}

Hence if we find an upper bound for the cardinality of $Z(\lambda_1,\dots,\lambda_{g-1})$, this is also an upper bound for $C(\Z_{(p)})_P$. There are several, sometimes ad hoc, ways of proving finiteness. We start by looking simply at calculations modulo $p$. We let $I$ denote the ideal inside $\Z_p\angle{x_1,\dots,x_r}$ generated by $\lambda_1,\dots,\lambda_{g-1}$, and $A$ the quotient $\Z_p\angle{x_1,\dots,x_r}/I$. Noting that $Z(\lambda_1,\dots,\lambda_{g-1})$ is then exactly $\Hom_{\Z_p}(A,\Z_p)$, we can use the following theorem, the statement and proof of which come from Theorem~4.12 of \cite{edixhoven20}.

\begin{prop}
\label{prop:finedix}
Let $\overline{A} = A/pA$. Assume $\overline{A}$ is finite. Then $\overline{A}$ is the product $\prod_{m \in \MaxSpec(\overline{A})} A_m$, and the sum of $\dim_{\F_p} \overline{A}_m$ over those $m\in \MaxSpec(\overline{A})$ with $\overline{A}/m = \F_p$ is an upper bound for $\Hom(A,\Z_p)$.
\end{prop}
\begin{proof}
We start by proving that $A$ is $p$-adically complete. This follows from the following more general fact: let $R$ be a Noetherian ring, and $I,J$ two ideals of $R$ such that $R$ is $J$-adically complete. We will then prove that $R/I$ is also $J$-adically complete. By Theorem 10.17 of \cite{atiyah}, any module over a complete ring injects into its completion, i.e. the map $R/I \to \widehat{R/I}$ is injective. Furthermore, as completion is exact, the surjection $R \to R/I$ gives rise to a surjection $R \to \widehat{R/I}$. The kernel contains $I$, so the map $R/I \to \widehat{R/I}$ is surjective and hence an isomorphism. Then we observe that $\Z_p\angle{x_1,\dots,x_r}$ is itself the $p$-adic completion of $\Z_p[x_1,\dots,x_r]$ and hence is complete, so $A$ is indeed $p$-adically complete.

Next, we use that $\overline{A}$ is a finite $\Z_p$-module, generated by the image of some finite set $S \subset A$. Using the series of exact sequences \[0 \to \overline{A} \xrightarrow{\cdot p^{n-1}} A/p^nA \to A/p^{n-1}A \to 0,\] we see by induction that $S$ indeed generates $A/p^nA$ as a $\Z_p$-module and by completeness, $S$ generates $A$ as well. Hence $A$ is a finite rank $\Z_p$-algebra.

Remember that a finite-dimensional algebra over a field is Artinian, and Artin rings are products of Artin local rings (Theorem 8.7 of \cite{atiyah}). So, we can write the $\F_p$-algebra $\overline{A}$ as a product of Artin local rings. As idempotents in $\overline{A}$ lift to $A$, we see this factorisation lifts, and we can decompose $A$ as
\[
A = \prod_{m \in \MaxSpec(\overline{A})} A_m.
\]
Then any morphism $A \to \Z_p$ factors through one of the $A_m$. If we tensor $A_m$ with $\Q_p$, we see that, after dividing out the nilradical, it is a product of fields. Hence $\Hom(A_m,\Z_p)$ is of cardinality at most $\text{rank}_{\Z_p}(A_m)$, and can be non-empty if and only if $A/m$ is isomorphic to $\F_p$. We conclude that the sum of $\dim_{\F_p} \overline{A}_m$ over those $m\in \MaxSpec(\overline{A})$ with $\overline{A}/m = \F_p$ is an upper bound for $\Hom(A,\Z_p)$
\end{proof}

To utilise this theorem, we only need to calculate the $\lambda_i$ modulo $p$. In our case, as all $\kappa_i$ and $f_i$ being linear modulo $p$ (as polynomials; they do not necessarily give linear maps), so are the $\lambda_i$, and we get as a special case the following corollary.
\begin{cor}
\label{cor:finedixlinear}
If the (not necessarily homogeneous) linear system of equations $\forall i: \lambda_i \equiv 0 \bmod p$ has respectively no or one solution, there is respectively none or at most one point in $C(\Z_{(p)})_P$.
\end{cor}

This leads to a now trivial proposition.
\begin{prop}
\label{prop:linchabauty}
If for all points $P \in C(\F_p)$, the linear system of $g-1$ equations modulo $p$ defining $C(\Z_p)_P \subset J(\Z_p)_t$ pulled back to $\overline{J(\Z_{(p)})_t}$ has $n_P \leq 1$ solutions, we get the inequality
\[
|C_{\Q}(\Q)| = |C(\Z_{(p)})| \leq \sum_{P \in C(\F_p)} n_P \leq |C(\F_p)|. 
\]
\end{prop}

Of note is that this upper bound for $|C(\Z_{(p)})|$ behaves agreeably when compared to the upper bound in Theorem 5.3 of \cite{poonen12}, which tells us that $|C(\Z_{(p)})| \leq |C(\F_p)| + 2g-2$ under the much lighter condition of $r < g, p > 2g$ where $p$ is still a prime of good reduction. To summarise: in Proposition~\ref{prop:linchabauty} we sacrifice certainty for being able to prove much sharper bounds. However, the loss of certainty is not too great; a simple heuristic analysis shows that the conditions are quite likely to be satisfied, as we shall see in the following lemma and proposition.

\begin{lem}
\label{lem:probindep}
Let $\F_q$ be any finite field, let $A$ be a random $(n+k)\times n$ matrix over $\F_q$, and $b$ randomly chosen from $\F_q^{n+k}$. Then the probability that $Ax=0$ has more than one solution is at most $\frac{1}{(q-1)q^k}$ and the probability that $Ax=b$ has more than one solution is at most $\frac{1}{(q-1)q^{2k+1}}$.
\end{lem}
\begin{proof}
Let $A_i$ denote the columns of $A$, let the space $V_i$ be the linear span of $A_1,\dots,A_i$ with $V_0 = 0$, and let $E_i$ denote the event that $A_i \in V_{i-1}$. Note that $Ax = 0$ has more than one solution if and only if $V_n$ is not $n$-dimensional, i.e. if and only if $E_1 \vee \cdots \vee E_n$ holds. As clearly for $1\leq i \leq n$ we have $\dim V_{i-1} \leq i-1$, we see that $\P(E_i)$, the probability of $E_i$ occurring,  is at most $q^{i-1-n-k}$. We then see
\begin{align*}
\P(E_1 \vee \cdots \vee E_n) &\leq \sum_{i=1}^n \P(E_i)\\
&\leq \sum_{i=1}^n q^{i-1-n-k}\\
&\leq \sum_{i=-\infty}^n q^{i-1-n-k}\\
&= \frac{1}{(q-1)q^k}.
\end{align*}
This shows that the probability of $Ax = 0$ having more than one solution is indeed at most $\frac{1}{(q-1)q^k}$. Furthermore, if $Ax = 0$ has more than one solution, then $\text{rank} A < n$ so there are at most $q^{n-1}$ possible values of $b$ such that $Ax = b$ has multiple solutions, hence the total probability is at most $\frac{1}{(q-1)q^{2k+1}}$.
\end{proof}
\begin{rem}
Using similar methods, one can show that $\P(E_1 \vee \cdots \vee E_n)$ is at least $\frac{1}{q^{k+1}}$, hence this lemma is asymptotically as sharp as possible.
\end{rem}

We now want to use this lemma to say something about the number of solutions to the linear systems of Proposition~\ref{prop:linchabauty}. Note that for a point $P \in C(\F_p)$ coming from a $\Z_{(p)}$-point, we always know there is at least one solution to the linear system; and indeed, it turns out that with a translation, one can assume the linear system is homogeneous. For all other points of $C(\F_p)$, we might assume the linear system modulo $p$ defining $C(\Z_p)_P$ pulled back to $\overline{J(\Z_{(p)})}$ is completely random. This motivates the assumptions of the following heuristic. 
\begin{prop}
\label{prop:probwork}
Assume that the map $C(\Z_{(p)}) \to C(\F_p)$ is injective. Assume that for points $P \in C(\F_p)$ coming from $C(\Z_{(p)})$ the linear system modulo $p$ defining $C(\Z_p)_P$ pulled back to $\overline{J(\Z_{(p)})_t}$ is a random homogeneous linear system, and for all other points, it is a random not-necessarily-homogeneous linear system. Assume furthermore that all these linear systems are independently random. Then for $p$ bigger than both $4g^2$ and $|C(\Z_{(p)})|$ the conditions of Proposition~\ref{prop:linchabauty} are not satisfied with probability at most
\[
(|C(\Z_{(p)})|+2p^{-(g-r-1)})(p-1)^{-1}p^{-(g-r-1)}.
\]
Hence for $r = g-1$ we expect a subset of primes with density $1$ to show finiteness of $C_\Q(\Q) = C(\Z_{(p)})$, and for $r < g-1$ we expect all but a finite number to work. These expectations are uniform in $g$ and $|C(\Q)|$.
\end{prop}
\begin{proof}
First note that our linear systems consist of $g-1$ equalities in $r$ variables, i.e. we can use Lemma~\ref{lem:probindep} with $k = g-1-r$. Let $a$ denote $|C(\Z_{(p)})|$, and let $b$ denote $|C(\F_p)|-a$. Under all our assumptions, the probability of none of the systems having more than one solutions is at least
\[
\left(1 - \frac{1}{(p-1)p^k}\right)^{a} \left(1 - \frac{1}{(p-1)p^{2k+1}}\right)^b.
\]
Using Bernoulli's inequality and the fact that $1- a\frac{1}{(p-1)p^k}$ and $1 - b\frac{1}{(p-1)p^{2k+1}}$ are positive, this is at least 
\[
\left(1- a\frac{1}{(p-1)p^k}\right)\left(1 - b\frac{1}{(p-1)p^{2k+1}}\right)
\]
which is itself at least $1 - a\frac{1}{(p-1)p^k} - b\frac{1}{(p-1)p^{2k+1}}$. Using the Hasse-Weil bound, we can make the estimate $b \leq 2p$. Then we can lower bound this by \[1 - a\frac{1}{(p-1)p^k} - 2\frac{1}{(p-1)p^{2k}} \geq 1 - (a+2p^{-k})\frac{1}{(p-1)p^k}.\]

For $r = g-1$ we have $k= 0$, and the probability of one of the conditions not being satisfied is at most $(a+2)/(p-1)$, hence we indeed expect an infinite but density $0$ subset of primes to fail. For $r < g-1$, we have $k > 0$; as $\sum_{p\text{ prime}} 1/p^2$ converges, one may expect a finite number of primes to fail.
\end{proof}
\begin{rem}
The author believes that this proposition gives a strong reason to expect the method of only computing modulo $p$ to work in practice; and if it does not, one can just take the next prime.
\end{rem}
\begin{rem}
\label{rem:probwork}
Consider the case $r=g-1$. As we expect a random $n\times n$-matrix to be invertible, we expect that if the conditions of Proposition~\ref{prop:linchabauty} are satisfied, we generally find an upper bound for $|C(\Q)|$ slightly lower than $|C(\F_p)|$. However, a not-necessarily-homogeneous linear system of dimension $(n+k)\times n$ is solvable with probability at most $p^{-k}$, as the dimension of the column space is at most $n$. Hence for $r = g-2$ we can expect to find an upper bound for $|C(\Q)|$ of approximately $|C(\Q)|+1$, and with probability approximately $(1-\frac{1}{p})^{|C(\F_p)|} \approx (1-\frac{1}{p})^p \approx e^{-1}$ even the optimal upper bound of $|C(\Q)|$. For $r = g-3$, we even expect to find an upper bound of $|C(\Q)|$ for a set of primes of density $1$; and for $r < g-3$, for all but a finite set of primes. 
\end{rem}

Now we briefly discuss what can be done if for a specific prime $p$, the conditions of Proposition~\ref{prop:linchabauty} are not satisfied. In general, even if $\overline{A}$ is not finite, we see $\Hom(A,\Z_p)$ factors through $\Z_p\angle{x_1,\dots,x_r}/(I:p)$ where $(I:p)$ is the saturation \[\{x \in \Z_p\angle{x_1,\dots,x_r} \mid \exists k \in \Z_{\geq 0}: p^kx \in I\}.\]
Then a higher precision calculation of the $\lambda_i$ can still lead to a succesful application of Proposition~\ref{prop:finedix}.

Another specific case is that of $r = 1$. In that case, we can use finite-precision approximations of $\lambda_1$ to deduce information about its Newton polygon and use that to bound the number of zeroes in $\Z_p$. We can even adapt this method if $r$ is bigger than $1$; sometimes it might be possible to use the implicit function theorem for power series, Lemma~\ref{lem:implicitfunction}, to write one of the variables as a power series in the other variables, and then substitute it in the other equations.

\begin{exmp}
If $r = g-1 = 2$ and \[\lambda_1 \equiv px-py,\lambda_2 \equiv x+y+pxy \bmod p^2,\] neither Proposition~\ref{prop:finedix} nor Newton polygons are instantly applicable. However, substituting $y = -x +px^2 \bmod p^2$ into the first equation gives the equation $2p(x-x^2) = 0 \bmod p^2$, or $x-x^2 = 0\bmod p$. Now both Proposition~\ref{prop:finedix} and Newton polygons give an upper bound of $2$ for $C(\Z_{(p)})_P$.
\end{exmp}

\section{Complications and improvements}
\label{section:remarks}
In this section, we quickly make some further remarks on how or when to apply the theory we have built up in previous sections to real cases.
\begin{itemize} 
\item Even if $r \geq g$, it might still be possible that $\overline{J(\Z_{(p)})}$ coincidentally has dimension $r'$ smaller than $r$ (for example, if $r > g$); then one can find a rank $r'$ subgroup of $J(\Z_{(p)})$ with the same closure inside $J(\Z_p)$, and do computations in this subgroup; we can expect this to work if $r' < g$. We can even use this to improve our calculations if we already have $r < g$, but $r'$ is even smaller. On the other hand, if $J(\Z_{(p)})$ is dense in $J(\Z_p)$, there is no way to apply Chabauty's method to the curve.
\item Note that in all of the discussion of the previous section, we are calculating upper bounds for the cardinality of $C(\Z_p)_P \cap \overline{J(\Z_{(p)})_t}$. It can and does however happen that this set is strictly bigger than $C(\Z_{(p)})_P$, as shown in \cite{Balakrishnan19}, where Coleman-Chabauty is used on a database of 16997 curves of genus $3$ with Mordell-Weil rank of the Jacobian equal to $1$. They treat several possible cases and examples where the set is bigger. One of the main families they have found can be given in the following way: let $Q$ be a $K$-point where $K$ is a quadratic number field where $p$ splits. Embedding $K$ into $\Q_p$, assume $Q-b$ lies in $J(\Z_p)_0$ and is torsion of order coprime to $p$ in $J(\Z_p)_0/J(\Z_{(p)})_0$. Then, as $J(\Z_p)$ is isomorphic as continuous group to $\Z_p^g$, the point $Q-b$ lies in $\overline{J(\Z_{(p)})_0}$, and clearly also in $C(\Z_p)$.
\item An important part of doing these calculations is working with the Mordell-Weil group $J(\Z_{(p)}) \cong J(\Q)$. Finding generators of $J(\Q)$ is as of yet a computationally difficult problem, and one may need to assume the Birch-Swinnerton-Dyer conjecture to even find the rank $r$. However, when one knows the rank, one does not necessarily need to find generators $J(\Q)$; if we can generate a subgroup of $J(\Q)$ of index finite and coprime to $p|J(\F_p)|$, it will have the same closure in $J(\Z_p)$. For this, we only need to generate sufficiently many independent points of the Jacobian and saturate subgroups with respect to some primes. Both of these tasks are easier than finding all of $J(\Q)$.
\item Even if this method does not work for a certain prime $p$, and neither for some other primes we tried, we can still use parts of the information we have gathered. For example, maybe the method tells us that we have found all points of $J(\Z_{(p)})$ stemming from $C(\Z_{(p)})$ except those in a certain fibre of the map $J(\Z_{(p)}) \to J(\F_p)$. We can then aggregate this information for different primes, choosing a smooth model of $C$ over for example $\Z[1/n]$ and lifting $J$ as well, by looking at the maps $J(\Z[1/n]) \to \prod_{p \in S} J(\F_p)$ for some set $S$ of primes of good reduction for $C_{\Q}$ not dividing $n$. This method is called the Mordell-Weil sieve, see for example \cite{stoll10} for an introduction.
\end{itemize}

\section{Implementations of linear Chabauty}\todo{Try to make this a bit more condensed.}
\label{section:explicit}
We now assume that our curve $C$ is hyperelliptic, i.e. given by the degree $2g+2$ homogenisation of an equation of the form
\[
y^2 = f(x)
\]
inside the weighted projective space $\P(1,g+1,1)$ where $f$ is a monic polynomial of degree $2g+1$ or $2g+2$. An alternative way of defining such a curve, and the one we will be using mainly, is as a glueing of two affine charts: $y^2 = f(x)$, and $w^2 = f^{r}(v)$, where $f^{r}(v)$ is the polynomial $v^{2g + 2} f(1/v)$, and a birational map between them is given by $(x,y) \mapsto (\frac{1}{x},\frac{y}{x^{g+1}})$. We also have the coordinates $X,Y,Z$ of $\P(1,g+1,1)$, with \[x = X/Z, y = Y/Z^{g+1}, v = Z/X, w = Y/X^{g+1},\] but beware; $\P(1,g+1,1)$ is not smooth and hence these coordinates do not behave nicely on all of $\P(1,g+1,1)$. We mainly use the first chart; we call any point that lies on it an affine point of $C$. Again for ease of exposition, we will treat the case that $f$ is monic of degree $2g+2$ (in general, one can demand $f$ has degree $2g+2$ by translating $f$ until the constant coefficient is non-zero, and then looking at $f^r$). In that case, near the line at infinity $C$ looks like $Y^2 = X^{2g+2}$, i.e. $(Y-X^{g+1})(Y+X^{g+1}) = 0$, and we see there are two points $\infty_+ = (1:1:0)$ and $\infty_- = (1:-1:0)$. Finally, we note that there is an involution on $C$ given by $\sigma(x,y) = (x,-y)$ and $\sigma(v,w) = (v,-w)$.

\subsection{Makdisi's algorithms}
We work in the Jacobian using Makdisi's representations for divisors. As we are using and adding on to an implementation by Mascot \cite{mascot18}, we briefly introduce his notation. This is a summary of Section~2.1 in \cite{mascot18}.

We first look at representing $J(k)$ where $k$ is a field. Given a divisor $D$ on $C$, denote
\[
\Lcal(D) = \{f \in k(C)^\times : \div(f) + D \geq 0 \} \sqcup \{0\}.
\]
We pick an effective divisor $D_0$ of degree $d_0 \geq 2g+1$; in the case of hyperelliptic curves, we will choose $(g+1)(\infty_+ + \infty_-)$. We set $V_n = \Lcal(nD_0)$. We let $n_Z$ be an integer $\geq 5d_0 + 1$, and assume, if necessary passing to an extension of $k$, that we have a set $Z$ of size $n_Z$ of distinct points in $C(k)$ outside the support of $D_0$; in fact, this will consist of affine points in our case. We have an evaluation map $V_5 \to k^Z$, evaluating a rational function at $Z$. By our choice of $n_Z$, this is an injective map, i.e. we can represent rational functions in $V_5$ by their values in $k^Z$. In this representation, we can add, subtract, or, if the degree at infinity is not too large, even multiply rational functions, by respectively adding, subtracting, or multiplying the corresponding vectors in $k^Z$. It is now also possible to represent subspaces of $V_5$ by giving a basis in $k^Z$. (Instead of passing to an extension of $k$, one could also evaluate functions on infinitesimal neighborhoods of $k$-points, i.e. compute Taylor expansions near those points.)

We now explain the representation of $J(k)$. Note that for any $x \in J(k)$, we have that $x + [D_0]$ is a divisor class of degree at least $2g+1$ and hence is equivalent to an effective divisor $E \geq 0$ of degree $d_0$. Then we represent $x$ by $\Lcal(2D_0 - E)$ inside $V_2$; by Riemann-Roch this is a $d_W$-dimensional subspace of $V_2$ where $d_W = d_0 + 1 - g$, and in particular we can represent it as a $n_z \times d_W$ matrix, itself representing a subspace of $k^Z$. This representation is nowhere near unique; there are many different effective divisors $E$ equivalent to $x + D_0$, and many bases for a subspace of $k^Z$.

As explained in Mascot's article, using this representation one can do all relevant computations in $J(k)$; adding, subtracting, finding the zero element, and very importantly: checking equality. Important from a computational standpoint is the complexity, which we write down in big O notation. As everything is simply linear algebra in spaces of dimensions $O(g)$, the complexity of all these operations, assuming calculations in the ground ring are $O(1)$, are all simply $O(g^\omega)$ where $\omega$ is the exponent of matrix multiplication.

\subsubsection{Going from \texorpdfstring{$\F_p$}{Fp} to \texorpdfstring{$\Z/p^e\Z$}{Z/peZ}}
We now know how to compute in $J(k)$ for $k$ a field such that $C(k)$ is big enough. In practice, if we want to calculate in $J(\F_p)$, this means passing to $J(\F_q)$ for some $q = p^a$ with $a$ large enough; by the Hasse-Weil bound this will work. However, for Chabauty we want to compute inside $J(\Z/p^e\Z)$. Luckily, Mascot's code takes care of this too, by passing from vector spaces over $\F_p$ to free $R$-submodules of $R^n$ with $R = \Z/p^e\Z$; in fact, all submodules of $R^n$ we will be seeing are free. That means all these submodules will have good reduction, i.e. they will remain free and of the same rank after tensoring with $\F_p$. If the maps between such modules also have good reduction, then all kernels, images, et cetera will also have these properties, and can first be calculated modulo $p$ using linear algebra, and then Hensel lifted modulo higher powers of $p$.

The final trick we need is extensions of $\Z/p^e\Z$. As said before, we need $n_Z$ affine points that are distinct modulo $p$, so we passed from $\F_p$ to an extension $\F_q$. The corresponding notion of an extension of $\Z/p^e\Z$ is given by taking an irreducible polynomial $\overline{T} \in \F_p[t]$ with $\F_q \cong \F_p[t]/\overline{T}$, arbitrarily lifting $\overline{T}$ to a polynomial $T \in \Z/p^e\Z[t]$, and looking at $R = (\Z/p^e\Z[t])/T$. Again, we will only be looking at free submodules of $R^n$, so we can again do normal linear algebra over $R \tensor \F_p = \F_q$, and using Hensel to lift.

\subsection{Implementing the Abel-Jacobi map and Mumford representations}
Now that we can do computations with elements in the Jacobian over $\Z/p^2\Z$, it only remains to construct elements in the Jacobian. Explicitly, we want to go from a degree zero divisor to an element in Mascot's representation. Most of the time these divisors are sums of points over the ring we are working with, but sometimes they are given as a so-called Mumford representation.
\begin{defn}
Let $C$ be any (hyper)elliptic curve given by $y^2 = f(x)$ with $f$ of degree $2g+2$ where $g$ is the genus of $C$. A Mumford representation is a pair $(a,b)$ with $a,b$ polynomials in $x$, representing the degree $0$ divisor $D(a,b)$ on $C$, given on the first affine chart by the equation $a(x) = 0, y = b(x)$ and on the line at infinity by $(-\deg a)\infty_+$. Such a pair $a,b$ is required to satisfy that
\begin{enumerate}
\item the polynomial $b$ is of degree at most $\deg a -1$;
\item the polynomial $a$ is monic of degree at most $g+1$;
\item the polynomial $a$ divides $b^2 - f$.
\end{enumerate}
\end{defn}
\begin{rem}
If $b$ does not satisfy the first condition, we can reduce $b$ modulo $a$. If $a$ does not satisfy the second condition, we can use the formula that on the first affine chart, we have
\[
D(a,b) + D\left(\frac{f-b^2}{a},b\right) = (y-b)
\]
meaning that with additional calculations of the behaviour of $(y-b)$ at infinity we can express $[D(a,b)]$ in the Jacobian in terms of $D(\frac{f-b^2}{a},b)$, and if $\deg a$ is strictly bigger than $g+1$, the degree of $\frac{f-b^2}{a}$ is at most $\deg a -2$.
\end{rem}

We will treat how to explicitly compute both the Abel-Jacobi embedding and divisors in Mumford representation in Makdisi's representation for the Jacobian. We start with the Abel-Jacobi embedding 
\begin{align*}
j_{\infty_+}: C &\to J \\
              P &\mapsto P-\infty_+.
\end{align*}
We will only need $j_{\infty_+}(P)$ and $j_{\infty_-}(P)$ for affine points $P$; as the calculation of $j_{\infty_-}(P)$ is entirely similar to $j_{\infty_+}(P)$, we only focus on $j_{\infty_+}(P)$. For this, we present the following algorithm:

\begin{algorithm}[H]
\label{alg:abeljacobi}
 \KwData{$C,J$, an affine point $P \in C(R)$ where $R = \Z/p^e\Z)$}
 \KwResult{A space of the form $\Lcal(2D_0 - E)$ where $E-D_0 = P - \infty_+$ as divisors and $E \geq 0$}
 $Z' \leftarrow Z \sqcup \{P\}$\;
 $B = (b_1,\dots,b_{g+3}) \leftarrow$ a basis of $\Lcal(D_0)$\; \label{alg:abeljacobi:basisLD0}
 \eIf{$(f^r)'(0) \not= 0$}{
   $F \leftarrow x^{g+1} + y$\;\label{alg:abeljacobi:defF1}
   }{
   $F \leftarrow x^{g+1} + x^g +  y$\;\label{alg:abeljacobi:defF2}
  }
 $b_{g+4} \leftarrow xF$\;
 $W \leftarrow$ a $(n_Z + 1)\times (g+4)$ matrix with rows being the evaluations of $B\sqcup{b_{g + 4}}$ on a point in $Z'$.\;
 $V \leftarrow \ker(\im W \subset R^{n_Z + 1} \to R)$, the projection on the last coordinate.\; \label{alg:abeljacobi:kerp}
 %Calculate $\Lcal(D_0 + \infty_+ - P)$\;
 $U \leftarrow \im(V \to R^{n_Z})$, where the last map is the projection on the first coordinates.\;
 Return $U$\;
 \caption{The Abel-Jacobi embedding}
\end{algorithm}

\begin{prop}
\label{prop:abeljacobi}
Algorithm~\ref{alg:abeljacobi} gives correct output.
\end{prop}
Before proving this proposition, we start with a quick lemma.
\begin{lem}
\label{lem:poles2d0i+}
The poles of $F$, as defined in line~\ref{alg:abeljacobi:defF1} or \ref{alg:abeljacobi:defF2}, are $g(\infty_+ + \infty_-) + \infty_+$. 
\end{lem}
\begin{proof}
We start by recalling that at the other affine patch, the curve $C$ is given by $w^2 = f^r(v)$ and by the assumption that $f$ is monic of degree $2g+2$ we have $f^r(0) = 1$. The points $\infty_\pm$ correspond to $(v,w) = (0,\pm 1)$ in this patch. Letting $g^r(v)$ be the polynomial $(f^r(v)-1)/v$, we can rewrite the equation for $C$ to $(w-1)(w+1) = vg^r(v)$. As the derivative of $(w-1)(w+1)$ to $w$ does not vanish at both of $\infty_\pm$, we see that $v$ is a uniformiser at both these points. That means that $v_{\infty_\pm}(x)$, the order of $x$ at $\infty_\pm$, is equal to $-1$.

Now, if $g^r(0)$ is non-zero, then $w- \pm 1$ is also a uniformiser at $\infty_\pm$ and non-zero at $\infty_\mp$, so \[(w+1)/v^{g+1} = y + x^{g+1}\] has poles exactly $(g+1)(\infty_+ + \infty_-) - \infty_-$ as we wanted to show.
And if $g^r(0)$ is zero, then $v_{\infty_\pm}(w-\pm1)$ is at least $2$ so $w-\pm 1 + v$ is a uniformiser at $\infty_\pm$ and non-zero at $\infty_\mp$, so \[(w+1+v)/v^{g+1} = y + x^{g+1} + x^{g}\] again has the right poles.
\end{proof}
\begin{rem}
\label{rem:complexityabel} Clearly, the complexity of Algorithm~\ref{alg:abeljacobi} is $O(g^\omega)$. For a point in the Jacobian that is represented as $\sum P_i - gb$, this gives an $O(g^{\omega+1})$ algorithm for computing it in Makdisi's representation.
\end{rem}

\begin{proof}[Proof of Proposition \ref{prop:abeljacobi}]
First note that by Riemann-Roch, the dimension of $\Lcal(D_0)$ is $g + 3$, and we also have by the proof of the previous lemma that $1,x,\dots,x^{g+1},y$ all lie in $\Lcal(D_0)$ and hence form a basis, so we can indeed find $B$ as in line~\ref{alg:abeljacobi:basisLD0}. Note that by Lemma~\ref{lem:poles2d0i+} the element $b_{g+4}$ lies in $\Lcal(D_0 + \infty_+)$ but not in $\Lcal(D_0)$, so as adding a point to a divisor causes the the dimension to increase by at most $1$, we have that $b_1,\dots,b_{g+4}$ is a basis for $\Lcal(D_0 + \infty_+)$; that it is in fact a basis is evident as this argument tells us it is a basis when tensored with $\F_p$.

Evaluating $\Lcal(D_0 + \infty_+)$ on $P$ gives a linear map $\Lcal(D_0 + \infty_+) \to R$, and the kernel is exactly $\Lcal(D_0 + \infty_+ - P)$; this is the resulting $U$ in line~\ref{alg:abeljacobi:kerp}. Furthermore we have the equality of divisors $P-\infty_+ = E - D_0$ where $E = P + D_0 - \infty_+ \geq 0$, so $\Lcal(D_0 + \infty_+ - P)$ is as a subspace of $V_2$ equal to $\Lcal(2D_0 - E)$. This last term is in fact in Mascot's representation, so this represents $P-\infty_+$ in the Jacobian.
\end{proof}

Now we move on to the Mumford representation. We present a way to go from the Mumford representation of a divisor to a Makdisi representation for the corresponding point on the Jacobian. This is based on private correspondence between Mascot and the author.
\begin{proof}[Algorithm/proof]
Let $(a,b)$ be the Mumford representation of a divisor $D$. Let $\deg a = d \leq g$, and denote $D_{\text{aff}}$ for the affine part of $D$. Note that $E-D_0 = D$ for $E = D_0 + D_{\text{aff}} - d\infty_+$, an effective divisor, so $\Lcal(2D_0 - E) = \Lcal(D_0 - D)$ is a Makdisi representation for $[D]$.

Of course, we can rewrite $\Lcal(D_0 -D)$ as 
\begin{multline*}
\Lcal((g+1)\infty_- + (g+1+d)\infty_+ - D_{\text{aff}}) \\= \Lcal(2D_0 - D_{\text{aff}} ) \cap \Lcal((g+1)\infty_- + (g+1+d)\infty_+).
\end{multline*}

We first calculate $\Lcal(2D_0 - D_{\text{aff}})$. Over $\F_p$ we have the equality
\begin{multline*}
\Lcal(2D_0 - D_{\text{aff}}) \\= a(x)\Lcal((2g+2-d)(\infty_- + \infty_+)) + (y-b(x))\Lcal((g+1)(\infty_- + \infty_+)).
\end{multline*}
Note all these Riemann-Roch spaces are of the form $H^0(C_{\F_p},\Fcal_{\F_p})$ where $\Fcal$ is an invertible $\O_{C_{\Z/p^k\Z}}$-module on $C_{\Z/p^k\Z}$ of degree higher than $2g-2$. Hence by Serre duality, $h^{1}(C_{\F_p},\Fcal_{\F_p}) = 0$. That means the base change map 
\[H^1(C_{\Z/p^k\Z},\Fcal) \tensor \F_p \to H^1(C_{\F_p},\Fcal_{\F_p}) = 0\]
is surjective, and by Theorem~III.12.11 of \cite{hartshorne} an isomorphism. Because $H^1(C_{\Z/p^k\Z},\Fcal)$ is a $\Z/p^k\Z$-module, this means it is $0$. Hence we can use Theorem~III.12.11b of \cite{hartshorne} to conclude that the base change map 
\[H^0(C_{\Z/p^k\Z},\Fcal) \tensor \F_p \to H^0(C_{\F_p},\Fcal_{\F_p})\]
is also an isomorphism, and $H^0(C_{\Z/p^k\Z},\Fcal)$ is a free $\Z/p^k\Z$-module. That means the equality 
\begin{multline*}
\Lcal(2D_0 - D_{\text{aff}}) \\= a(x)\Lcal((2g+2-d)(\infty_- + \infty_+)) + (y-b(x))\Lcal((g+1)(\infty_- + \infty_+)).
\end{multline*}
also holds over $\Z/p^k\Z$. Note that both the Riemann-Roch spaces on the right hand side can be easily calculated, as they are generated by the monomials in $x,y$ of the right order at $\infty_{\pm}$.

For $\Lcal((g+1)\infty_- + (g+1+d)\infty_+)$, we first inductively calculate the spaces $W_n := \Lcal(2D_0 - n\infty_-)$. For $W_1$ we have a basis \[1,x,\dots,x^{2g+1},y,xy,\dots,x^g y, x^{g+1}y+x^{2g+2}\], and for $n \geq 0$ we have by a simple use of Lemma 2.2 of \cite{makdisi04} that for $1 \leq n \leq g+1$ we have \[W_{n} = \{s \in \Lcal(2D_0) \mid s\Lcal(2D_0) \in W_1 \cdot W_{n-1}\}.\]
This is a purely linear constraint, hence simple linear algebra allows us to calculate all $W_n$ for $n \leq g+1$. Finally, we have
\[
\Lcal((g+1)\infty_- + (g+1+d)\infty_+) = \{ s \in \Lcal((g+1+d)(\infty_+ + \infty_-)) \mid sx^{g+1-d} \in W_d).
\] 
Once again noting that we can write down an explicit basis for the space $\Lcal((g+1+d)(\infty_+ + \infty_-))$, we can now calculate $\Lcal((g+1)\infty_- + (g+1+d)\infty_+)$ and finish the computation.
\end{proof}

\subsection{Parameters at \texorpdfstring{$J$}{J}}
\label{subs:paramj}
Being able to calculate in the Jacobian, we can move on to the final ingredient for explicit computations: parameters at points of $J$. For any proper, smooth curve $C$ over $\Z_{(p)}$, with a $\Z_{(p)}$-point $b$, we have a birational map from $C^{(g)}$, the $g$-fold symmetric product of $C$, to $J$, given on points by sending $[(P_1,\dots,P_g)]$ to $[\sum P_i - gb]$. This map is \'etale at $[(P_1,\dots,P_g)]$ if $\Lcal(P_1 + \cdots + P_g)$ has dimension $1$ as the fibre of $P_1 + \cdots + P_g -gb$ is exactly $\P \Lcal(P_1 + \cdots + P_g)$. Also, if all $P_i$ are distinct, then the map $C^g \to C^{(g)}$ is \'etale at $P_1,\dots,P_g$ as well. Then finding parameters $t_1,\dots,t_g$ at $Q := [\sum_{i=1}^g P_i - gb]$ just comes down to finding parameters at each of the $P_i$. In our case of a hyperelliptic curve, this is just an easy computation; for a point $(a,b)$ on the first affine chart, we can take $x-a$ if $b$ is non-zero and $y$ if $a$ is zero. We can also compute the inverse of the map $C(\Z/p^k\Z)_{P_i} \to p\Z/p^k\Z$; this is just a simple exercise in Hensel lifting.

This now means $J(\Z_p)_Q$ is parametrised as the product of $C(\Z_p)_{P_i}$. In particular, we have a bijection
\[
\prod_{i=1}^g C(\Z/p^k\Z)_{P_i} \to J(\Z/p^k\Z)_{Q}.
\] As we are able to explicitly compute this map using the Abel-Jacobi map, and we are able to test equality in $J(\Z/p^k\Z)$, we can compute the map $(t_1,\dots,t_g): J(\Z/p^k\Z)_Q \to (p\Z/p^{k}\Z)^g$ by first computing its inverse and storing all found values. This gives an algorithm for computing this map that consists of $O(p^{(k-1)g})$ operations in the Jacobian.

\begin{rem}
\label{rem:faster}
We can do this faster at the cost of consistency, by applying another algorithm by Mascot, Algorithm~7 in \cite{mascot18}. This algorithm computes a rational map $J \to \P V_2$, dependent on the choice of some specific degree effective divisors, with complexity $O(g^\omega)$. Of course, it is trivial to compute parameters in $\P V_2$, but the downside is that the rational map may not need to be defined where we want it to be. In that case, one can just make another choice of effective divisors and construct a different rational map; remember that we only need the map to be defined in our residue disc of choice.
\end{rem}

\subsection{Interpolating polynomials}
\label{subs:interpolate}
The $\lambda_i$ from Theorem~\ref{thm:final} are in general not computationally available, as they are power series, but we can approximate them modulo powers of $p$. For example, we know that the $\lambda_i$ are linear modulo $p$. And in particular, if $f_1,\dots,f_{g-1}$ are as in Remark~\ref{rem:fislinear}, then $\lambda_i = \kappa_i$ for $i \leq g-1$, so they are also of degree at most $m$ modulo $p^m$ for $m < p-1$. Knowing this, we can compute $\lambda_i$ modulo powers of $p$ by interpolation, and the following formula, coming from \cite{salzer}: if $f$ is a polynomial over a ring $R$ of degree $m$ in $n$ variables and $m!$ is a unit in $R$, then we have the equality of polynomials
\[
f(x_1,\dots,x_n) = \sum_{i_1 + \cdots + i_n \leq m} f(i_1,\dots,i_n) \binom{m-x_1-\cdots-x_n}{m-i_1-\cdots-i_n} \prod_{j=1}^n \binom{x_j}{i_j}.
\]
Note that this formula requires evaluation of $f$ at $\binom{m+n}{n}$ points; this is clearly optimal, as there are $\binom{m+n}{n}$ monomials of degree at most $m$. Also note that for $m=1$, the complexity of calculating this formula is simply $O(n)$ applications of $f$, and $O(n)$ calculations in $R$.

Note that evaluating $\lambda_i$ at a point in $J(\Z_{(p)})_t$ comes down to calculating the value of a parameter at that point, and this is something we can do. That means we can explicitly compute $\lambda_i$ modulo $p^m$ for $m < p-1$. Using Proposition~\ref{prop:finedix} or another way, we can hope to find an upper bound on the number of common zeroes of the $\lambda_i$, and hence an upper bound on $C(\Z_{(p)})_P$.

This formula for interpolation unfortunately does not work for $m \geq p$; then the values of $\lambda_i$ on $\F_p^{n}$ do not determine $\lambda_i$, and one needs to evaluate on extensions on $\F_q^n$ and look for more general interpolation formulas (e.g., Lagrange interpolation).

\subsection{Complexity}
We assume necessary data for the Mordell-Weil group of the Jacobian is given; that is, we have a set of generators for either $J(\Z_{(p)})$, or just a full rank subgroup of index coprime to $p|J(\F_p)|$. We also assume that all relevant elements in $J(\Z_{(p)})$ and $J(\F_p)$ are in the subgroup generated by $C(\Z_{(p)})$ and $C(\F_p)$. Finally, we assume that we can find rational maps $J \to \P V_2$ as in Remark~\ref{rem:faster}.

Then by Remark~\ref{rem:complexityabel} and Section~\ref{subs:interpolate}, we compute $\lambda_i \bmod p$ in $O(rg^{\omega+1})$; then solving the linear system of equations $\forall i \lambda_i \equiv 0 \bmod p$ has complexity lower than that, and hence checking if $|C(\Z_{(p)})_P|$ is at most $1$ can be done in $O(rg^{\omega+1})$. Repeating this for all $\F_p$-points, gives by the Hasse-Weil bound a complexity of $O\left((p + g\sqrt{p})rg^{\omega+1}\right)$.

\section{An explicit example}
\label{section:example}
We treat the hyperelliptic curve $C_\Q/\Q$ with first affine chart given by
\[
y^2 = f(x) = x^6+ 8x^5+ 22x^4+ 22x^3+ 5x^2+ 6x+ 1.
\]
This curve has genus $2$, and the Mordell-Weil group is isomorphic to $\Z$. This curve is also treated in Example~8.2 of \cite{poonen12}.

We denote the standard involution on this curve by $\sigma$. It has six known rational points $\infty_+,\infty_- = \sigma(\infty_+)$ and $\theta = (0,1),\sigma(\theta),\eta=(-3,1),\sigma(\eta)$. We take $p = 5$, and use the same equation for the model $C$ over $\Z_{(5)}$. It turns out that $C(\F_5)$ has seven points; the reductions of the rational points, and $(1,0)$. Again taking the map $C \to J$ to be subtraction of $\infty_+$ on points, all seven points in $C(\F_5)$ lie in the image of $J(\Z_{(p)})$.

We will first show there is at most one point in the same residue disc as $\theta$. As we already know there is at least one point, namely $\theta$, we can simplify the calculations by looking at the map $j_{\theta}: C \to J$ to be subtraction of $\theta$. Letting $\theta_{\mu}$ for $\mu \in \F_p$ denote the deformations of $\theta$ in $C(\Z/p^2\Z)_{\theta}$, with $\theta_\mu \mapsto \mu$ being a parameter at $\theta$, we see the image of $C(\Z/p^2\Z)_\theta$ in $J(\Z/p^2\Z)_0$ is $\{\theta_\mu - \theta \mid \mu \in \F_p\}$. Also, identifying $J(\Z_{(p)})_0$ with $\Z^r$, the image of $J(\Z_{(p)})_0$ in $J(\Z/p^2\Z)_0$ can be given as a map $\F_p^r \to J(\Z/p^2\Z)_0$. Choosing parameters by giving a local chart $J \to \P(V_2)$, we want to exactly show that the two maps $C(\Z/p^2\Z)_\theta \to J(\Z/p^2\Z)_0, \F_p^r \to J(\Z/p^2\Z)_0$ together form a map $\F_p \oplus \F_p^r \to J(\Z/p^2\Z)_0$ with kernel $0$. Now that this is a statement in linear algebra, it can be verified easily, and it turns out that indeed the residue disc of $C(\Z_{(p)})$ containing $\theta$ contains only $\theta$.

Similarly, it follows that all points in $C(\F_5)$ lift to at most one $\Z$-point. Noting that the group generated by $\sigma$ acts on $C(\Z_{(p)})$ and has no fixed points, this means $|C(\Z_{(p)})| \leq 6$ so our list of rational points is complete.

Code for the calculations in this section can be found in \cite{spelier2020}.

\newpage
\bibliographystyle{alpha}
\addcontentsline{toc}{section}{References}
\bibliography{ref.bib}

\end{document}
