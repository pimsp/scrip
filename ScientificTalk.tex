\documentclass[15pt]{beamer}
\usepackage[T1]{fontenc}
\usepackage{multicol}
\usepackage{ragged2e}   %new code
\usepackage[utf8]{inputenc}
\usepackage[brazil]{varioref}
\usepackage[square,sort,comma,super,authoryear]{natbib}
\usepackage{xmpmulti}
\usepackage{epsfig}
\usepackage{subcaption}
\captionsetup{compatibility=false}
\usepackage{ru,graphicx,hyperref,url} % 
\usepackage{amsthm}
\usepackage{eucal}
\usepackage{amssymb}
%\usepackage{mdsymbol}
\usepackage{mathrsfs}
\usepackage{dsfont}
%\usepackage[usenames,dvipsnames,svgnames,table]{xcolor}
\usepackage{tikz,tikz-cd}
\usepackage{wrapfig,lipsum,booktabs}
\usepackage{caption}
\usepackage[ruled,linesnumbered]{algorithm2e}
\usepackage[all]{nowidow}

%\usepackage{hyperref}
\usepackage{amstext} % for \text macro
\usepackage{array}   % for \newcolumntype macro
\usepackage{mathtools}

% Enable colored hyperlinks
\hypersetup{colorlinks=true}

% The following three lines are for crossmarks & checkmarks
\usepackage{pifont}% http://ctan.org/pkg/pifont
\newcommand{\cmark}{\ding{51}}%
\newcommand{\xmark}{\ding{55}}%

\newcommand{\den}{\mathop{\mathgroup\symoperators den}\nolimits}
\newcommand{\ggd}{\mathop{\mathgroup\symoperators ggd}\nolimits}
\newcommand{\des}{\text{d.e.s.d.a.\ }}
%\newcommand{\opg}{\text{d.e.s.d.a.\ }}
\newcommand{\blokje}{\hfill $\Box$\\}
%Om je bewijs of je antwoord af te sluiten.
\newcommand{\header}[1]{\vspace{0.5cm}\framebox[\linewidth]{\textsc{Exercise #1}}\\}
\newcommand{\deel}[1]{\textbf{#1)}\ }
\newcommand{\macht}[1]{\mathcal{P}(#1)}
\newcommand{\entier}[1]{\left\lfloor #1 \right\rfloor}
\newcommand{\A}{\mathbb{A}}
\newcommand{\N}{\mathbb{N}}
\newcommand{\Z}{\mathbb{Z}}
\newcommand{\G}{\mathbb{G}}
\newcommand{\C}{\mathbb{C}}
\newcommand{\Q}{\mathbb{Q}}
\newcommand{\vx}{\mathbf{x}}
\newcommand{\vy}{\mathbf{y}}
\newcommand{\vz}{\mathbf{z}}
\newcommand{\Fcal}{\mathcal{F}}
\newcommand{\Lcal}{\mathcal{L}}
\renewcommand{\O}{\mathcal{O}}
\renewcommand{\P}{\mathbb{P}}
\newcommand{\Pcal}{\mathcal{P}}
\newcommand{\NP}{\mathcal{NP}}
\newcommand{\NPC}{\mathcal{NPC}}
\newcommand{\F}{\mathbb{F}}
\renewcommand{\angle}[1]{\hspace{-2pt}\left\langle #1 \right\rangle}
\newcommand*\diff{\mathop{}\!\mathrm{d}}
\newcommand{\rad}{\text{rad}}
\newcommand{\x}{x}
\newcommand{\m}{\mathfrak{m}}
\newcommand{\tensor}{\otimes}
\newcommand{\Ring}{\textbf{Rings}}
\newcommand{\Sets}{\textbf{Sets}}

\DeclareMathOperator{\mon}{mon}
\DeclareMathOperator{\Hom}{Hom}
\DeclareMathOperator{\Gal}{Gal}
\DeclareMathOperator{\End}{End}
\DeclareMathOperator{\sgn}{sgn}
\DeclareMathOperator{\trdeg}{trdeg}
\DeclareMathOperator{\rk}{rk}
\DeclareMathOperator{\Fun}{Fun}
\DeclareMathOperator{\im}{im}
\DeclareMathOperator{\id}{id}
\DeclareMathOperator{\sm}{sm}
\DeclareMathOperator{\pr}{pr}
\DeclareMathOperator{\diag}{diag}
\DeclareMathOperator{\Spec}{Spec}
\DeclareMathOperator{\MaxSpec}{MaxSpec}
\DeclareMathOperator{\Spf}{Spf}
\DeclareMathOperator{\sep}{sep}
\DeclareMathOperator{\Ann}{Ann}
\DeclareMathOperator{\GL}{GL}
\DeclareMathOperator{\Mat}{Mat}
\DeclareMathOperator{\Jac}{Jac}
\let\S\undefined
\DeclareMathOperator{\S}{S}
\let\div\undefined
\DeclareMathOperator{\div}{div}


\addtobeamertemplate{block begin}{}{\justifying}
\setbeamertemplate{section in toc}[sections numbered]
\AtBeginSection[]
{
  \begin{frame}{Table of Contents}
  \begin{multicols}{2}
      \tableofcontents[currentsection]
    \end{multicols}
  \end{frame}

}

% The title of the presentation:
%  - first a short version which is visible at the bottom of each slide;
%  - second the full title shown on the title slide;

\title[Geometric linear Chabauty]{Geometric linear Chabauty}

% Optional: a subtitle to be dispalyed on the title slide
% \subtitle{Show where you're from}

% The author(s) of the presentation:
%  - again first a short version to be displayed at the bottom;
%  - next the full list of authors, which may include contact information;
\author[Pim Spelier]{
  Pim Spelier\\\medskip
  {\small {spelier.pim@gmail.com}
  }}

% The institute:
%  - to start the name of the university as displayed on the top of each slide
%    this can be adjusted such that you can also create a Dutch version
%  - next the institute information as displayed on the title slide
\institute[Leiden University]{
  Mathematical Institute \\
  }

% Add a date and possibly the name of the event to the slides
%  - again first a short version to be shown at the bottom of each slide
%  - second the full date and event name for the title slide
\date{\today}

\begin{document}

\begin{frame}
  \titlepage
\end{frame}

%\begin{frame}{Table of Contents}
%    \begin{multicols}{2}
%    \tableofcontents
%    \end{multicols}
%\end{frame}

% Section titles are shown in at the top of the slides with the current section 
% highlighted. Note that the number of sections determines the size of the top 
% bar, and hence the university name and logo. If you do not add any sections 
% they will not be visible.
\section{Introduction}
\begin{frame}
    \frametitle{Introduction}
    \framesubtitle{Problem}
    \begin{block}{Problem}
        Let $C_\Q$ a proper, smooth, geometrically connected curve over $\Q$ of genus $g \geq 2$. Find $C_\Q(\Q)$.
    \end{block}
\end{frame}

\begin{frame}
    \frametitle{Introduction}
    \framesubtitle{Classical solutions}
    $C_\Q$ a proper, smooth, geometrically connected curve over $\Q$ of genus $g \geq 2$.
    \begin{block}{Faltings's theorem (1983)} 
        $C_\Q(\Q)$ is finite.
    \end{block}\pause
    \begin{itemize}[<+->]
	\item This is not effective.
	\item Can be made effective by Chabauty.
	\end{itemize}
\end{frame}

\begin{frame}
    \frametitle{Introduction}
    \framesubtitle{Chabauty}
    $C_\Q$ a proper, smooth, geometrically connected curve over $\Q$ of genus $g \geq 2$.\\
    Let $J_\Q$ be the Jacobian of $C_\Q$, $p$ a prime of good reduction, $r$ the rank of Mordell-Weil group $J_\Q(\Q)$, $b\in C_\Q(\Q)$ and $j_b: C_\Q \to J_\Q$ subtracting $b$ on points.\\ \pause
    \[
\begin{tikzcd}[ampersand replacement=\&]
C_\Q(\Q) \arrow[d] \arrow[r] \& J_\Q(\Q) \arrow[d] \\
C_\Q(\Q_p) \arrow[r]              \& J_\Q(\Q_p)        
\end{tikzcd}
\]
\vspace{50pt}
\end{frame}

\begin{frame}
    \frametitle{Introduction}
    \framesubtitle{Chabauty}
    $C_\Q$ a proper, smooth, geometrically connected curve over $\Q$ of genus $g \geq 2$.\\
    Let $J_\Q$ be the Jacobian of $C_\Q$, $p$ a prime of good reduction, $r$ the rank of Mordell-Weil group $J_\Q(\Q)$, $b\in C_\Q(\Q)$ and $j_b: C_\Q \to J_\Q$ subtracting $b$ on points.\\ 
    \[
\begin{tikzcd}[ampersand replacement=\&]
\& C_\Q(\Q) \arrow[d] \arrow[r] \& \overline{J_\Q(\Q)} \arrow[d] \arrow[r, no head] \& \dim r\\
\dim 1 \arrow[r, no head] \& C_\Q(\Q_p) \arrow[r]              \& J_\Q(\Q_p)   \arrow[r, no head]  \& \dim g   
\end{tikzcd}
\]\pause
    \begin{block}{Definition}
        Let $C_\Q(\Q_p)_1 = C_\Q(\Q_p) \cap \overline{J_\Q(\Q)} \supset C_\Q(\Q)$.
    \end{block}\pause
\end{frame}

\begin{frame}
    \frametitle{Introduction}
    \framesubtitle{Chabauty}
    $C_\Q(\Q_p)_1 = C_\Q(\Q_p) \cap \overline{J_\Q(\Q)}$.
    \begin{block}{Chabauty's theorem (1941)}
        If $r < g$, then $C_\Q(\Q_p)_1$ is finite (and hence so is $C_\Q(\Q)$).
    \end{block}\pause
    Classically, Coleman-Chabauty (1985): find $1$-forms on $J_{\Q_p}$ whose Coleman integral vanish on the Mordell-Weil group $J_{\Q}(\Q)$, and pull back to $C_\Q$ to make effective.
\end{frame}

\begin{frame}
    \frametitle{Recent developments}
    \framesubtitle{Kim-Chabauty and quadratic Chabauty}
    \begin{itemize}[<+->]
        \item Coleman has been generalised by Kim (2009), by replacing $J_\Q$ with Selmer variety.
        \item Gives infinite chain\[ C_\Q(\Q_p)_1 \supset C_\Q(\Q_p)_2 \supset \cdots \supset C_\Q(\Q).\]
        \item Quadratic Chabauty focuses on $C_\Q(\Q_p)_2$, which is finite if $r < g + \rho - 1$ where $\rho$ is the N\'eron-Severi rank of $J_\Q$
        \item Used by Balakrishnan, Dogra, M\"uller, Tuitman, Vonk (2019) to calculate rational points of $X_s(13)$.
    \end{itemize}
\end{frame}

\begin{frame}
    \frametitle{Recent developments}
    \framesubtitle{Geometric quadratic Chabauty}
    Edixhoven and Lido replace $J_\Q$ by $T$, another geometric object, and spread out over $\Z$.\\\pause
    In contrast with Coleman-Chabauty, pull back equations for $C(\Z_p)$ inside $T(\Z_p)$ to $T(\Z)$. \\\pause
    Question: what happens when we pull apply this to $J$, the linear case?\\\pause
    Aim of thesis: describe what happens in the linear case +  heuristics + how to implement + example.
\end{frame}

\section{Setting and strategy}
\begin{frame}
    \frametitle{Setting}
    $C/\Z_{(p)}$ proper, smooth curve with generic fibre $C_\Q$, with $p > 2$.\\\pause
    $J/\Z_{(p)}$ relative Jacobian, with embedding $j_b: C \to J$, subtracting $b \in C(\Z_{(p)})$.\\\pause
    $P \in C(\F_p)$ with $t = P - b \in J(\F_p)$ lifting to a $\Z_{(p)}$-point.\\\pause
    Consider the diagram of residue discs:\[
    \begin{tikzcd}[ampersand replacement=\&]
        C(\Z_{(p)})_P \arrow[d] \arrow[r] \& J(\Z_{(p)})_t \arrow[d] \\
        C(\Z_p)_P \arrow[r]              \& J(\Z_p)_t        
    \end{tikzcd} \]
\end{frame}

\begin{frame}
    \frametitle{Strategy}
    \[
    \begin{tikzcd}[ampersand replacement=\&]
        C(\Z_{(p)})_P \arrow[d] \arrow[r] \& J(\Z_{(p)})_t \arrow[d] \\
        C(\Z_p)_P \arrow[r]              \& J(\Z_p)_t        
    \end{tikzcd} \]
    Strategy:
    \begin{itemize}[<+->]
        \item Describe $C(\Z_p)_P$ and $J(\Z_p)_t$
        \item Describe the inclusion $C(\Z_p)_P \to J(\Z_p)_t  $
        \item Describe the map $\kappa: J(\Z_{(p)})_t \to J(\Z_p)_t$
    \end{itemize}
\end{frame}

\section{Method}
\begin{frame}
    \frametitle{$\Z_p$-points in a residue disc}
    Let $X$ be a smooth $\Z_p$-scheme of relative dimension $n$. Let $x \in X(\F_p)$.\\\pause
    Then $O_{X,x}$ has maximal ideal generated by $p$ and parameters $t_1,\dots,t_n$.
    \begin{block}{Fact (generalisation of Hensel's Lemma)}
    Evaluating $t = (t_1,\dots,t_n)$  gives a bijection $X(\Z_p)_x \to p\Z_p^n$.
    \end{block}\pause
    Dividing by $p$ gives $X(\Z_p)_x \xrightarrow{~} \Z_p^n$.
\end{frame}

\begin{frame}
    \frametitle{Abel-Jacobi map $C \to J$}
    Hence $C(\Z_p)_P \to J(\Z_p)_t$ induces a map $\Z_p \to \Z_p^g$.
    \begin{definition}
    The ring $\Z_p\angle{x_1,\dots,x_n}$ consists of those elements of $\Z_p[[x_1,\dots,x_n]]$ converging on all of $\Z_p^n$.
    \end{definition}
    \begin{block}{Fact}
    There are $f_1,\dots,f_{g-1} \in \Z_p\angle{x_1,\dots,x_{g}}$ whose zero-set cut out the image of $\Z_p \to \Z_p^g$. These are linear modulo $p$ and when the parameters are chosen correctly, are linear.
    \end{block}\pause
\end{frame}

\begin{frame}
    \frametitle{Inclusion Mordell-Weil group $J(\Z_{(p)})$}
    What about $J(\Z_{(p)})_0 \to J(\Z_p)_0$? This is $\kappa: (\Z^r,+) \to (\Z_p^g,\oplus)$.\\\pause
    Group structure on $J(\Z_p)_0 \cong p\Z_p^g$ given by formal group $F \in \Z_p[[\vx,\vy]]^g$.\pause
    This induces a formal logarithm.
\end{frame}

\begin{frame}
    \frametitle{Logarithm}
    \begin{block}{Proposition}
    The formal logarithm for the formal group corresponding to $J(\Z_p)_0$ induces an isomorphism $\log_p: J(\Z_p)_0 = (\Z_p^g,\oplus) \to (\Z_p^g,+)$ with inverse $\exp_p$, with both $\exp_p,\log_p \in \Z_p\angle{x_1,\dots,x_{g}}^g$ linear mod $p$ (and also well behaved mod higher powers).
    \end{block}\pause
    \begin{block}{Theorem}
    The map $\kappa: \Z^r \to \Z_p^g$ is given by $g$ convergent power series, linear mod $p$, and can hence be extended to $\kappa: \Z_p^r \to \Z_p^g$ with $\im \kappa = \overline{J(\Z_{(p)})_0}$
    \end{block}\pause
    Proof:\[
    \begin{tikzcd}[ampersand replacement=\&]
{(\Z^d,+)} \arrow[rr, "\kappa"] \arrow[rd, "\log_p \circ \kappa"] \&                                   \& {(\Z_p^g,\oplus)} \\
                                                                  \& {(\Z_p^g,+)} \arrow[ru, "\exp_p"] \&                  
\end{tikzcd}\]
\end{frame}

\begin{frame}
    \frametitle{Final theorems}
    \begin{definition}
        Let $\lambda_1,\dots,\lambda_{g-1} = \kappa^* f_1,\dots,\kappa^* f_{g-1} \in \Z_p\angle{x_1,\dots,x_r}$, i.e. the pullback to $\overline{J(\Z_{(p)})_P}$ of the equations defining $C(\Z_p)_P$ inside $J(\Z_p)_t$. Then $Z(\lambda_1,\dots,\lambda_{g-1}) \subset \Z_p^r$ surjects onto $C(\Z_p)_{1,P} = C(\Z_p)_P \cap \overline{J(\Z_{(p)})_t}$. \pause I.e., with $A = \Z_p\angle{x_1,\dots,x_r}/(\lambda_1,\dots,\lambda_{g-1})$, then $|C(\Z_{(p)})_P| \leq |\Hom(A,\Z_p)| = |Z(\lambda_1,\dots,\lambda_{g-1})|$
    \end{definition}\pause
    \begin{block}{Proposition (Edixhoven-Lido)}
    If \[\overline{A} = \F_p[x_1,\dots,x_r]/(\overline{\lambda_1},\dots,\overline{\lambda_{g-1}})\] is finite, then $|\Hom(A,\Z_p)| \leq |\dim_{\F_p} \overline{A}|$
    \end{block}\pause
    Proof: $A$ is complete.\\\pause
    Important: our $\lambda_i$ are linear mod $p$.
\end{frame}

\begin{frame}{Final theorems}
    \begin{definition}
        Let $\lambda_1,\dots,\lambda_{g-1} = \kappa^* f_1,\dots,\kappa^* f_{g-1} \in \Z_p\angle{x_1,\dots,x_r}$, i.e. the pullback to $\overline{J(\Z_{(p)})_P}$ of the equations defining $C(\Z_p)_P$ inside $J(\Z_p)_t$. Then $Z(\lambda_1,\dots,\lambda_{g-1}) \subset \Z_p^r$ surjects onto $C(\Z_p)_{1,P} = C(\Z_p)_P \cap \overline{J(\Z_{(p)})_t}$. I.e., with $A = \Z_p\angle{x_1,\dots,x_r}/(\lambda_1,\dots,\lambda_{g-1})$, then $|C(\Z_{(p)})_P| \leq |\Hom(A,\Z_p)| = |Z(\lambda_1,\dots,\lambda_{g-1})|$
    \end{definition}
    \begin{block}{Proposition (Single digit precision geometric linear Chabauty)}
    If for all points $P \in C(\F_p)$, the linear system $\overline{\lambda_1}=\dots=\overline{\lambda_n} = 0$ defining $C(\Z_p)_P \subset J(\Z_p)_t$ pulled back to $\overline{J(\Z_{(p)})_t}$ has $n_P \leq 1$ solutions, we get the inequality
    \[
    |C_{\Q}(\Q)| = |C(\Z_{(p)})| \leq \sum_{P \in C(\F_p)} n_P \leq |C(\F_p)|. 
    \]
    \end{block}
\end{frame}

\section{Heuristics}
\begin{frame}
    \frametitle{Modelling}
    Recall: for each point $P$ in $C(\F_p)$, we have an affine linear system of $g-1$ equations $\overline{\lambda_i} = 0$ in $r$ variables.\\\pause
    Note: $\F_p[x_1,\dots,x_r]/(\overline{\lambda_1},\dots,\overline{\lambda_{g-1}})$ is finite $\Leftrightarrow \overline{\lambda_1} = \cdots = \overline{\lambda_{g-1}} = 0$ has 0 or 1 solution.\\\pause
    If $P$ comes from a $\Z_{(p)}$ point, we model as random homogenous.\\\pause
    Otherwise we assume it to be model as affine.\\\pause
    Question: under these assumptions, what is the probability that our method gives a (good) upper bound?
\end{frame}

\begin{frame}
    \frametitle{Random linear system}
    \begin{block}{Lemma}
    Let $A \in \Mat_{(n+k)\times n}(\F_q)$ be random. Then the probability of $A$ having non-trivial kernel is at most $\frac{1}{(p-1)p^k}$
    \end{block}\pause
    \begin{proof}
    Denote the columns by $A_0,\cdots,A_{n-1}$, let $V_i = \angle{A_0,\dots,A_{i-1}}$ and denote $E_i$ for the event $A_i \in V_i$. Then 
    \begin{align*}
    \P(\ker A \not= 0) &= \P(E_0 \vee \cdots \vee E_{n-1}) \\
                       &\leq \sum_{i=0}^{n-1} \P(E_i) \\
                       &\leq \sum_{i=0}^{n-1} p^{i-(n+k)} \\
                       &\leq \frac{1}{(p-1)p^k}.
    \end{align*}
    \end{proof}
\end{frame}

\begin{frame}
    \frametitle{Heuristic}
    \begin{block}{Lemma}
    Let $A \in \Mat_{(n+k)\times n}(\F_q)$ be random. Then the probability of $A$ having non-trivial kernel is at most $\frac{1}{(p-1)p^k}$
    \end{block}
    \begin{block}{Proposition}
    Under all of our assumptions (+ some technical assumptions), single-digit-precision calculations don't work with probability at most\[
    (|C(\Z_{(p)})|+2p^{-(g-r-1)})(p-1)^{-1}p^{-(g-r-1)}.
    \]
    \end{block}\pause
    \begin{block}{Corollary}
    We expect our method to work for a set of primes with density $1$; if $r < g-1$, for all but finitely many primes.
    For $r < g-1$, we expect to find with probability at least $1/e$ the clearly optimal upper bound of $|C(\Z_{(p)})|$ for $|C(\Z_{(p)})|$.
    \end{block}
\end{frame}

\section{Implementation}
\begin{frame}
    \frametitle{Implementation}
    \framesubtitle{General remarks}
    Parameters at $C$: easy.\\ \pause
    Parameters at $J$: use birationality with $C^{(g)}$.\\ \pause
    Problem: we need to easily represent elements of $J(\Z/p^2\Z)$\\\pause
    Solution: Makdisi's representation
\end{frame}

\begin{frame}
    \frametitle{Implementation}
    \framesubtitle{Makdisi's representation}
    Let $k$ be a field, and $C/k$ a curve.\\\pause
    Let $D_0$ be an effective divisor of degree $d_0 \geq 2g+1$.\\\pause
    Let $V = \Lcal(5D_0)$. Let $S$ be a set of points of size $\geq 5d_0 + 1$ on $C$ not meeting $D_0$, and note $V \hookrightarrow k^S$. \\\pause
    Let $[x] \in J(k)$, pick $E\geq 0$ with $[x] = [D_0 - E]$, and represent $[x]$ by $\Lcal(2D_0 - E) \subset V \subset k^S$.\\\pause
    This is non-unique, but equality can be tested. \\\pause
    Also addition, subtraction, $\dots$. All linear algebra inside $V \subset k^S$.\\\pause
    For $k = \F_p$ this lifts easily to $\Z/p^2\Z$; everything is a free $\Z/p^2\Z$-module inside $(\Z/p^2\Z)^S$, hence still just linear algebra.\\\pause
    Hence all operations in $O(g^\omega) $ where $\omega = 2.37...$.
\end{frame}

\begin{frame}    
    \frametitle{Implementation}
    \framesubtitle{Complexity}
    \begin{block}{Proposition}
        If $C$ is hyperelliptic, the Abel-Jacobi map $C \to J$ can be computed in $O(g^\omega)$.
    \end{block}\pause
    
    \begin{block}{Proposition}
    	Assume we have a set of $r$ elements of $J(\Z_{(p)})$ that generate a subgroup of index finite and coprime to $p|J(\F_p)|$ (+ other technical assumptions). Then the complexity of executing single-digit-precision geometric linear Chabauty on hyperelliptic curves is $O(|C(\F_p)|rg^{\omega+1})$.
    \end{block}\pause
\end{frame}

\begin{frame}
    \frametitle{Example}
    \begin{block}{Example}
        \[C\colon y^2 = x^6 + 8x^5 + 22x^4 + 22x^3 + 5x^2 + 6x + 1\]
        with $g = 2, r = 1$ and $|C_\Q(\Q)_{\text{known}}| = 6$. Choose $p = 5$. We find $|C(\Z_{(5)})| \leq 7$. By the hyperelliptic involution having two fixed points, $|C_\Q(\Q)|$ is even and hence $C_\Q(\Q)_{\text{known}} = C_\Q(\Q)$.
    \end{block}
\end{frame}


\end{document}

